\chapter{Propuesta Pedagógica}

En este capítulo pasamos a detallar la ventaja pedagógica que se pretende obtener en en base a los objetivos que presentamos en el segundo capítulo y a los antecedentes vistos en el tercero.

\section{Análisis general del problema}

La enseñanza de la programación suele considerarse bastante complicada, ``ya que muchos estudiantes encuentran bastantes dificultades cuando empiezan a programar'' (\cite{rubio_uso_2018}), de hecho ``La existencia de altas tasas de fracaso y la subsiguiente incapacidad de los estudiantes para escribir programas simples al final de una unidad de programación son solo dos de los problemas que inciden en las facultades de informática todo el mundo'' (\cite{bruce_contemporary_nodate}).

\bigskip
Aun así la programación es uno de los pilares mas importantes de la informática, por lo que su aprendizaje es de carácter obligatorio para todo el mundo. De hecho en muchos países como Finlandia, Alemania y Estonia promueven el aprendizaje de programación desde la educación primaria, en nuestro país El estudio de las CC en Educación Primaria y Secundaria se encuentra en su fase inicial y se ha
empezado a introducir recientemente por lo que aún no ha sido adoptado por la mayoría de centros escolares. Como resultado, el número de niños y niñas que a día de hoy estudian ciencias de la computación es
todavía una minoría (\cite{fecyt_educacion_2016}) pero se está promoviendo el aprendizaje a edades cada vez más tempranas.

\bigskip
Para enseñar programación a nuestros alumnos tenemos que tener en cuenta las dificultades que suponen la interacción ordenador/alumno y ``contar con una actitud receptiva a los problemas que les plantea a los alumnos la introducción de un lenguaje formal y su uso para programar y resolver problemas'' \cite{vitale_psycopedagogical_1990}.

\section{Análisis de las soluciones existentes}

Tras analizar las tres soluciones libres descritas en los antecedentes (Coderunner, Python Tutor y Jupyter Notebooks) hemos encontrado que las tres tienen el mismo problema. En todos los casos nuestros alumnos tienen que aprender a usar una herramienta concreta que posiblemente no vayan a volver a utilizar en su futuro profesional. Esto sería comparable a lo que didácticamente decimos que el alumno ``aprende a aprobar exámenes'' mas que aprender los contenidos de la asignatura. Por esto, nos gustaría proponer una metodología donde los alumnos usen herramientas que van a tener que usar en futuro profesional (o académico) como pueden ser un sistema de control de versiones así como sistemas de integración continua, o \textit{continuous integration (CI)}.

\section {Temario relacionado con la programación informática}

Tal y como indica el Decreto 110-2016. Las tecnologías de la información y de la comunicación para el aprendizaje y el conocimiento se utilizarán de manera habitual como herramientas integradas para el desarrollo del currículo. Asimismo en los currículos de las distintas ramas educativos donde podemos ejercer como profesores incluyen asignaturas relacionadas con la informática y de forma más específica con la programación informática.

\subsection {Bachillerato}

En la Orden de 14 de julio de 2016 se especifica el currículo de

Según la legislación Tecnologías de la Información y de la Comunicación \textbf{FALTAAAAAA}


FALTA TEXTO

\subsection {Ciclo Formativo Grado Medio}

La Orden EDU/2187/2009, de. 3 de Julio establece el currículo del ciclo formativo de Grado Medio correspondiente al título de ``Técnico en Sistemas Microinformáticos y Redes (SMR)'' que es el único ciclo de grado medio relacionado con la informática donde la única asignatura ligeramente relacionada con la programación es ``Aplicaciones web'', pero dicha asignatura se centra más en el despliegue de aplicaciones que en la programación en sí, por lo que aunque nuestra metodología se podría adaptar para enseñar a configurar como desplegar aplicaciones web no entra dentro del alcance de este trabajo por lo que se deja como posible trabajo futuro.

\subsection {Ciclo Formativo Grado Superior}

Existen tres ciclos formativos de grado superior del plan LOE relacionados con la informática, la Orden EDU/392/2010, de 20 de enero, por la que se establece el currículo del ciclo formativo de Grado Superior correspondiente al título de ``Técnico Superior en Administración de Sistemas Informáticos en Red'' describe las asignaturas ``Lenguajes de Marca y Sistemas de Gestión de Información'' y ``Gestión de Bases de Datos'' en las que nuestra metodología se puede aplicar para corregir los ejercicios ya que se enseñan los lenguajes HTML, SQL, Javascript, XML así como algunos subconjuntos de XML como son DTD y XSD. Los currículos del título de ``Técnico Superior en Desarrollo de Aplicaciones Web'' definido en la Orden EDU/2887/2010, de 2 de noviembre y el de ``Técnico Superior en Desarrollo de Aplicaciones Multiplataforma'' definido en la Orden EDU/2000/2010, de 13 de julio definen varias asignaturas, algunas comunes, como pueden ser ``Programación'', ``Bases de Datos'' y ``Lenguajes de marcas y sistemas de gestión de información'' y otras específicas de cada título como pueden ser ``Programación multimedia y dispositivos móviles'' o ``Desarrollo web en entorno cliente'' que están centradas en el aprendizaje de la programación informática y en los que si encajaría nuestra metodología.


\subsection {Formación Profesional Básica}

La Orden ECD/1030/2014, de 11 de junio define el ``Título Profesional Básico en Informática y Comunicaciones'' y la Orden ECD/1633/2014, de 11 de septiembre define el  ``Título Profesional Básico en Informática de Oficina'', siendo ambas titulaciones las únicas relacionadas con la informática de todas las que componen la formación profesional básica, pero al igual que con el ciclo formativo de Grado Medio ambos currículos carecen de temario específico de programación por lo que queda fuera del ámbito de este proyecto.


\section{Encuesta de percepción}

Se realizó una breve encuesta y se difundió entre algunos profesores de diferentes ámbitos con el fin de ver si nuestra hipótesis de que la corrección de ejercicios era la parte mas tediosa que realizaban como docentes.

\bigskip
Aprovechamos la encuesta para obtener también algunos datos interesantes sobre la forma de trabajar de los docentes, de cara a trabajos futuros. La encuesta se puede consultar en la siguiente dirección web: \url{https://forms.gle/xxSz6mQDjswo5B137}.

\bigskip
La encuesta estuvo activa desde el 19 hasta el 24 de Mayo de 2019 y se recopilaron un total de 46 respuestas. Puede parecer una población pequeña pero se compartió ``de boca a oreja'' indicando que por sólo lo compartieran con otros profesores a ser posible relacionados con la informática para no desvirtuar las respuestas.


\begin{enumerate}

\item \textbf{¿Dónde sueles impartir tus clases?}

Con esta pregunta queríamos tener una visión de la población encuestada, ya que al difundirla entre profesores y animarles a compartirla entre sus compañeros profesores era muy posible que sus red de contactos no se limitara a profesores de Educación Secundaria, así que dejamos elegir entre las siguientes opciones:

\begin{itemize}
    \item Educación Primaria
    \item Educación Secundaria
    \item Universidad (Grado/Máster)
    \item Practicas MAES
\end{itemize}

\bigskip
La gran mayoría de los encuestados han resultado ser profesores de Educación Secundaria, por lo que los resultados han sido satisfactorios, los siguientes han sido estudiantes en prácticas en el \master los cuales podríamos englobar dentro del conjunto de los de Secundaria, llegando así casi al 90\% de la población encuestada. Le siguen la respuestas de algunos profesores de Universidad y por último algunos maestros de Educación Primaria.

\begin{figure}[H]
\centering
\includegraphics[width=1.0\textwidth]{../images/quiz_1}
\caption{¿Dónde sueles impartir tus clases?}
\label{fig:quiz_1}
\end{figure}


\item \textbf{Valora las siguientes tareas que sueles realizar}

Esta es quizá la pregunta más importante de toda la encuesta, porque queríamos saber la opinión de diversos profesores sobre las tareas de corrección de ejercicios y/o exámenes. Además nos ha servido para hacernos una idea de lo que opinan de las tareas más comunes de un profesor, había que valorarlas según se consideraban ``divertidas'', ``normales'' o ``tediosas'' entendiéndose tediosas como aburridas. Las opciones eran las siguientes:

\begin{itemize}
    \item Preparación de material
    \item Preparación de clases
    \item Impartir clases
    \item Corrección de ejercicios
    \item Evaluación final
    \item Tutorías
\end{itemize}

Como podemos ver, la hipótesis de que corregir ejercicios resulta tedioso ha sido validada. De hecho ha sido la única de las opciones que no ha recibido ningún voto como actividad ``divertida''. Del resto de preguntas podemos deducir que a la mayoría de profesores les divierte impartir clases, esto es algo normal. No tendría sentido una respuesta diferente dado el carácter vocacional de la profesión, y si nos centráramos en los aspectos económicos es mas flagrante en el mundo de la informática porque los sueldos de la empresas privadas están muy por encima de lo que se puede llegar a cobrar como docente. El resto de preguntas se han considerado normales, habiendo opiniones en las tres vertientes. Quizá remarcar que la evaluación es la que menos votos de tarea ``divertida'' han recibido. Aunque esto puede variar mucho de un centro a otro.

\begin{figure}[H]
\centering
\includegraphics[width=1.0\textwidth]{../images/quiz_2}
\caption{Valora las siguientes tareas que sueles realizar}
\label{fig:quiz_2}
\end{figure}

\item \textbf{¿Elaboras tus propios ejercicios?}

En esta pregunta queríamos saber si los profesores elaboran sus propios ejercicios. En mi experiencia como alumno así como en mi experiencia como docente en prácticas he visto exactamente lo que se ve reflejado en la encuesta, hay profesores que elaboran todo su material, los hay que lo hacen de forma regular pero no siempre, también los hay que los realizan de forma ocasional. Aquí me gustaría remarcar que la opción ``Nunca'' ha recibido muy pocos votos cuando en mi experiencia también hay una mucho docentes que utilizan el material de otros profesores en lugar del suyo propio.

\begin{figure}[H]
\centering
\includegraphics[width=1.0\textwidth]{../images/quiz_3}
\caption{¿Elaboras tus propios ejercicios?}
\label{fig:quiz_3}
\end{figure}

\item \textbf{¿Crees que hay que compartir los ejercicios con el resto de la comunidad?}

Esta pregunta era para ver la repercusión que tiene a nivel docente los conceptos del software libre y las licencias abiertas como la GPL y la Creative Commons así como la Wikipedia y otros proyectos que abogan por la difusión libre y gratuita tanto de software como de contenidos educativos.

\bigskip
Por nuestra experiencia pensábamos que no iba a haber tan buena acogida del ``Si'' como ha habido, pero nos ha sorprendido gratamente que la gran mayoría de los profesores encuestados apoyen el compartir el material de apoyo. También es algo que se puede considerar normal ya que hoy en día es difícil hacer uso de recursos de terceros encontrados online para apoyar nuestras clases. De hecho en mi periodo como docente en prácticas me sirvió para ver que al contrario que en mi época de estudiante, donde usábamos el libro de texto, los docentes usan recursos como Kahoot o YouTube para hacer las clases mucho mas dinámicas.

\begin{figure}[H]
\centering
\includegraphics[width=1.0\textwidth]{../images/quiz_4}
\caption{¿Crees que hay que compartir los ejercicios con el resto de la comunidad?}
\label{fig:quiz_4}
\end{figure}

\item \textbf{¿Sueles entregar correcciones de los ejercicios?}



\begin{figure}[H]
\centering
\includegraphics[width=1.0\textwidth]{../images/quiz_5}
\caption{¿Sueles entregar correcciones de los ejercicios?}
\label{fig:quiz_5}
\end{figure}




\item \textbf{¿Qué opinas de los sistemas de corrección automática?}



\begin{figure}[H]
\centering
\includegraphics[width=1.0\textwidth]{../images/quiz_6}
\caption{¿Qué opinas de los sistemas de corrección automática?}
\label{fig:quiz_6}
\end{figure}


\item \textbf{Sugerencias para mejorar las clases}

Incorporamos a la encuesta un apartado para que los profesores aportaran sus sugerencias para mejorar las clases, a continuación vamos a poner los datos recopilados en bruto en dicho apartado:

\begin{itemize}
    \item Hacerlas dinámicas y participativas, huir de la explicación meramente
    \item Lo más eficiente para el aprendizaje del alumnado es que ellos mismos corrijan, así si se les indican los errores suelen verlos y evitarlos después
    \item Que parte de la evaluación la hagan los propios alumnos (son más conscientes de los errores cometidos).
    \item Fomentar el trabajo basado en proyectos colaborativos
    \item Correcciones en clase de la totalidad de los ejercicios, haciendo partícipes a los alumnos.
    \item Coevaluacion, aula invertida. Cooperativo
    \item Participación continuada de alumno
    \item La tipología de ejercicio más eficiente a la hora de su corrección es el tipo test. Existe una optimización de tiempo de corrección bastante notable en comparación con ejercicios con preguntas de desarrollo, donde se tarda más por no tener tan automatizada la respuesta correcta (hay que leer la respuesta), la puntuación se somete a matices por estar incompleta, en ocasiones es difícil comprender la letra del alumno, etc.
\end{itemize}


\end{enumerate}

\section{Metodología propuesta}


\subsection{Objetivos}

El objetivo de nuestra metodología es enseñar programación de una forma diferente, con ejercicios, autocorrección y con tecnologías que se usan en el día a día de cualquier informático que se dedique profesionalmente a ello. Para ello FALTA TEXTO

\subsection{Contenidos}

\subsection{Competencias}

\subsection{Temporización}




