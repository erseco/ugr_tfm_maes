\chapter{Antecedentes}

En este capítulo vamos analizar el estado de arte actual y las tecnologías candidatas a utilizarse en este proyecto en base a los objetivos que presentamos en el capítulo anterior.

\section {Procesos de verificación de código}


\subsection {Desarrollo guiado por pruebas (TDD)}

El desarrollo guiado por pruebas de software, o Test-Driven Development (TDD) es una práctica de ingeniería de software que involucra otras dos prácticas: Escribir las pruebas primero (Test First Development) y Refactorización (Refactoring). Para escribir las pruebas generalmente se utilizan las pruebas unitarias (unit test en inglés). En primer lugar, se escribe una prueba y se verifica que las pruebas fallan. A continuación, se implementa el código que hace que la prueba pase satisfactoriamente y seguidamente se refactoriza el código escrito. El propósito del desarrollo guiado por pruebas es lograr un código limpio que funcione. La idea es que los requisitos sean traducidos a pruebas, de este modo, cuando las pruebas pasen se garantizará que el software cumple con los requisitos que se han establecido.

\subsection {Desarrollo guiado por comportamiento (BDD)}

El desarrollo guiado por comportamiento, o Behavior-Driven Development (BDD) es una práctica de ingeniería de software que surgió a partir del desarrollo guiado por pruebas (TDD). El desarrollo guiado por el comportamiento combina las técnicas generales y los principios del TDD junto con ideas de análisis y diseño orientado a objetos para proveer al desarrollo de software y a los equipos de administración de herramientas compartidas y un proceso compartido de colaboración en el desarrollo de software.


\section {Herramientas de verificación de código}

\subsection {Pytest}

pytest es un framework de test unitarios para el lenguaje de programación Python que facilita la creación de pruebas sencillas y escalables. Las pruebas son expresivas y legibles y no se requieren código adicional.

\subsection {RSpec}

RSpec es una herramienta de tests BDD que sirve para probar código escrito en el lenguaje de programación Ruby. Se utiliza ampliamente en aplicaciones de producción. La idea básica detrás de este concepto es la de que las pruebas se escriben primero y el desarrollo se basa en escribir solo el código suficiente para cumplir con esas pruebas.



\section {Herramientas interactivas de corrección de código}

Existen diversas herramientas de corrección de codigo que funcionan de forma interactiva, vamos a enumerar las principales:

\subsection {Coderunner}

CodeRunner es un plug-in gratuito de código abierto para Moodle que puede ejecutar código de programa escrito por los estudiantes en respuesta a una amplia gama de preguntas de programación en muchos lenguajes diferentes. Está pensado principalmente para su uso en cursos de programación informática, aunque puede utilizarse para calificar cualquier pregunta cuya respuesta sea texto. Normalmente se usa en el modo de prueba adaptativa de Moodle. Los estudiantes escriben su código en respuesta a cada pregunta de programación y pueden ver los resultados de su caso de prueba inmediatamente. Luego pueden corregir su código y volver a enviarlo, normalmente por una pequeña penalización.

Se puede obtener desde \url{https://coderunner.org.nz}

\subsection {Python Tutor}

Python Tutor es una herramienta de visualización de programas basada en web para Python, ya que dicho lenguaje se está convirtiendo en un lenguaje popular para la enseñanza de cursos introductorios de programación. Con esta herramienta, los profesores y los alumnos pueden escribir programas de Python directamente en el navegador web (sin instalar ningún complemento), avanzar y retroceder a través de la ejecución para ver el estado de las estructuras de datos en tiempo de ejecución y compartir sus visualizaciones de programas en la web.

En los últimos tres años, más de 200,000 personas han usado Python Tutor para visualizar sus programas \cite{GuoSIGCSE2013}. Además, los instructores de más de una docena de universidades como UC Berkeley, MIT, la Universidad de Washington y la Universidad de Waterloo lo han utilizado en sus cursos de ciencias de la computación. Python Tutor es un software gratuito y de código abierto, se puede obtener desde \url{http://pythontutor.com}.


\subsection {Jupyter Notebooks}

Jupyter Notebook (anteriormente IPython Notebooks) es un entorno basado en la web para la creación de documentos interactivos. El término "cuaderno" puede hacer referencia coloquialmente a muchas entidades diferentes, principalmente la aplicación web Jupyter, el servidor web Jupyter o el formato de documento Jupyter dependiendo del contexto. Un documento Jupyter Notebook es un documento JSON, siguiendo un esquema versionado, y conteniendo una lista ordenada de celdas de entrada/salida que pueden contener código, texto (usando Markdown), matemáticas, gráficos y medios enriquecidos, normalmente terminando con la extensión ".ipynb".

El proyecto Jupyter está licenciado bajo la BSD por lo que uso y distribución es gratuito, se puede obtener desde \url{https://jupyter.org}

\subsection {Turingscraft's CodeLab}

http://www.turingscraft.com/


\section {Sistemas de control de versiones}

\subsection {GIT}

\section {Sistemas de integración continua (CI)}

\subsection {Travis-CI}

\subsection {Circle}

\subsection {BitBucket Pipelines}
