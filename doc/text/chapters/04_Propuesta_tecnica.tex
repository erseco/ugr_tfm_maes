\chapter{Propuesta Técnica}

En este capítulo vamos a definir y a detallar los procesos que vamos a seguir para solucionar el problema planteado en base a los objetivos que presentamos en el segundo capítulo ya los antecedentes vistos en el tercero.

\section{Análisis general del problema}

La enseñanza de la programación suele considerarse bastante complicada (referencia rubio).

Aun así la programación es uno de los pilares mas importantes de la informática, por lo que su aprendizaje es de carácter obligatorio para todo el mundo. De hecho en muchos países como "" se está promoviendo su aprendizaje desde edades tempranas (referencia).

\section{Herramientas necesarias}

\subsection{Sistemas de control de versiones}

Lo primero que necesitaremos es enseñar a nuestros alumnos a utilizar un sistema de control de versiones, aunque existen varios sistemas (SVN, Mercurial, CVS), hoy en día estándar de facto es Git que además es software libre.

Para una administración más sencilla de nuestro código utilizaremos un cliente web como es GitHub, aunque existen otros como pueden ser GitLab o BitBucket.

\subsubsection {Git}

Git es un sistema de control de versiones distribuido para el seguimiento de los cambios en el código fuente durante el desarrollo de software. Está diseñado para coordinar el trabajo entre programadores, pero se puede utilizar para realizar un seguimiento de los cambios en cualquier conjunto de archivos.  Sus objetivos incluyen la velocidad, la integridad de los datos y la compatibilidad con flujos de trabajo distribuidos y no lineales.

Git fue creado por Linus Torvalds en 2005 para el desarrollo del núcleo de Linux.

Como con la mayoría de los sistemas de control de versiones distribuidos, y a diferencia de la mayoría de los sistemas cliente-servidor cada directorio de Git es un repositorio completo con un historial completo y capacidades de seguimiento de versiones completas, independiente del acceso a la red o a un servidor central.

Git es software libre y de código abierto distribuido bajo los términos de la Licencia Pública General GNU versión 2.

\subsubsection {GitHub}

GitHub es un servicio de alojamiento basado en web para el control de versiones a través de Git, fue adquirido por MicroSoft en 2018. Se utiliza sobre todo para código fuente. Ofrece toda la funcionalidad de control de versiones distribuido y gestión de código fuente de Git, además de añadir sus propias características.

Proporciona control de acceso y varias funciones de colaboración como seguimiento de errores, pull requests, gestión de tareas y wikis para cada proyecto. Además cuenta con un sub-proyecto llamado GitHub Education que es ampliamente utilizado como herramientas para la formación de programadores (\cite{hernandez_integracion_2018}).


\section {Sistemas de integración continua (CI)}

\subsubsection {Travis-CI}

Travis CI es un servicio alojado de integración continua que se utiliza para construir y testear proyectos de software alojados en GitHub.

Los proyectos de código abierto pueden ser probados de forma gratuita a través del sitio travis-ci.org. En cambio los proyectos privados deben pagar para poder ser ejecutados a través del sitio travis-ci.com.

Gran parte del código fuente de Travis CI es software libre y está disponible en GitHub.

\subsection{Lenguajes utilizados}

Al igual que no existe una única solución a un problema, existen multitud de lenguajes de programación, nosotros nos vamos a centrar en solamente cuatro de ellos y vamos a explicar los motivos:

\subsubsection{Python}

Python es un lenguaje de programación interpretado cuya filosofía hace hincapié en una sintaxis que favorezca un código legible. Fue desarrollado por Guido van Rossum en 1991 y hoy en día es uno de los lenguajes de programación más utilizados en cursos de introducción a la programación ya que su sintaxis suele ser más sencilla, asemejándose al 'pseudocódigo".

Según el (REFERNCIA A ESTUDIO STACKOVERFLOW) Python es uno de los lenguajes con mayor proyección de futuro y que del que mas se ha incrementado su uso, además (REFERENCIA A DJANGO-GIRLS), referencia a estudios

Basándonos en las recomendaciones del Python Consortium (REFERENCIA A LA RECOMENDACIÓN DE LA PYTHON CONSORTIUM) vamos a centrar nuestros ejercicios en Python3 ya que no tiene sentido enseñar a nuestros alumnos Python2 pues además de desaconsejado, este lenguaje va a perder su soporte en un futuro muy cercano (REFERENCIA AL FIN DE SOPORTE DE PYTHON2).

\subsubsection{Ruby}

Ruby es un lenguaje de programación creado por el programador japonés Yukihiro Matsumoto en 1995. Es un lenguaje fuertemente orientado objetos ya que todos los tipos de dato son a su vez un objeto.

\bigskip
Ruby empezó a ganar popularidad tras la publicación del framework de aplicaciones web Ruby On Rails (RoR) por parte de David Heinemeier Hansson en 2005 ya que el mismo simplificaba muchísimo el desarrollo siguiendo el patrón Modelo Vista Controlador (MVC).

\bigskip
Debido a su sencilla sintaxis y su fuerte orientación a objetos es ampliamente utilizado para enseñar las particularidades de dicho paradigma.

\subsubsection{C}

El lenguaje de programación C es un lenguaje de propósito general desarrollado por Dennis Ritchie en los Laboratorios Bell entre 1969 y 1972, es el lenguaje de programación más popular para crear software para sistemas embebidos y micro-controladores, aunque también se puede utilizar para crear aplicaciones.

\bigskip
Es un lenguaje tipado estáticamente (en tiempo de compilación) y considerad de de medio nivel ya que dispone de las estructuras típicas de los lenguajes de alto nivel pero permitiendo un control a muy bajo nivel. Los compiladores suelen ofrecer extensiones al lenguaje que posibilitan mezclar código en ensamblador con código C, acceder a memoria o controlar diferentes dispositivos.

\subsubsection{HTML}

El lenguaje de marcas de hipertexto o HTML por sus siglas en inglés, es un lenguaje de marcado para la elaboración de páginas web. La primera versión del lenguaje fue creada por Tim Berners-Lee en 1991 mientras trabajaba en el Centro Europeo de Investigaciones Nucleares (CERN) en Suiza. Actualmente sus especificaciones están a cargo del World Wide Web Consortium (W3C).

\bigskip
HTML no es un lenguaje de programación, es un lenguaje de marcado que sirve para definir documentos estandarizados.

\section{Metodología de aprendizaje}

Algunas de las herramientas necesarias para nuestro sistema de corrección requieren algo de formación previa, por lo que vamos a definir una metodología de aprendizaje para las mismas

\subsection{Aprendiendo a usar Git}

Como ya explicamos anteriormente Git es uno de los sistemas de control de versiones más utilizados en la actualidad. Es dificil encontrar hoy en día una empresa que se dedique a la programación de forma profesional que no la utilice. A pesar de ello no forma parte del currículo de los ciclos formativos de informática. De hecho, salvo contadas excepciones, tampoco se aprende su uso en los Grados en Ingeniería Informática. De ahí la ventaja de nuestro sistema, ya que vamos a aprovechar para enseñarles a utilizar una herramienta que van a usar de forma exhaustiva durante su futura carrera profesional.

\bigskip
Para aprender a utilizar Git nos vamos a basar en diversos manuales que enseñan a utilizarlo de una forma sencilla (\cite{popov_control_2012}) sin necesidad de tener conocimientos de programación.

\subsubsection{Introducción a Git}

Un sistema de control de versiones es una herramienta que registra los cambios realizados sobre un conjunto de archivos a lo largo del tiempo, de modo que se pueden volver a un estado previo en cualquier momento. Aunque se usa de forma mayoritaria para código fuente  se pueden guardar versiones de cualquier tipo de archivo.

Las principales características de un sistema de control de versiones son revertir archivos a un estado previo, revertir el proyecto entero a un estado anterior, comparar cambios a lo largo del tiempo, ver quién modificó por última vez algo que puede estar causando un problema así como saber quién introdujo un error y cuándo.

Un método de control de versiones usado por mucha gente es copiar los archivos a otro directorio (indicando la fecha y hora en que lo hicieron, si son avispados). Este enfoque es muy común porque es muy simple, pero también tremendamente propenso a errores. Es fácil olvidar en qué directorio te encuentras, y guardar accidentalmente en el archivo equivocado o sobrescribir archivos que no querías.

En cambio Git es un sistema de control de versiones distribuido, los clientes no sólo descargan la última instantánea de los archivos: replican completamente el repositorio. Así, si un servidor muere, y estos sistemas estaban colaborando a través de él, cualquiera de los repositorios de los clientes puede copiarse en el servidor para restaurarlo (figura \ref{fig:dvcs}).

\begin{figure}[h!]
\centering
\includegraphics{../images/dvcs}
\caption{Diagrama de control de versiones distribuido}
\label{fig:dvcs}
\end{figure}

\subsubsection{Creando una cuenta en GitHub}

Para crear una cuenta en GitHUb debemos acceder a su página web \url{https://github.com} y hacer click en SignUp (figura \ref{fig:git1}).

\begin{figure}[h!]
\centering
\includegraphics[width=1.0\textwidth]{../images/git1}
\caption{Pagina de registro de GitHub}
\label{fig:git1}
\end{figure}

\subsubsection{Creando nuestro primer repositorio}

Crear un repositorio en GitHub es muy intuitivo, solo tenemos que hacer clic en el botón New y rellenar los datos que nos solicita (figura \ref{fig:git2}). De igual forma podemos crear un repositorio de forma local en nuestro ordenador ejecutando el comando \texttt{git init}
\begin{figure}[h!]
\centering
\includegraphics[width=1.0\textwidth]{../images/git2}
\caption{Ventana de creación de repositorio en GitHub}
\label{fig:git2}
\end{figure}

\subsubsection{Comandos básicos de Git}

Aquí podemos ver algunos de los comandos más habituales de Git, se puede encontrar la referencia completa en su documentacion oficial en \url{https://git-scm.com/doc}.

\begin{itemize}
  \item \textbf{git clone}: Clona un repositorio
  \item \textbf{git status}: Nos dice el estado de un repositorio
  \item \textbf{git commit}: Nos permite guardar los cambios en una rama
  \item \textbf{git checkout}: Nos permite cambiar de rama
  \item \textbf{git branch}: Nos permite crear y listar ramas
  \item \textbf{git push}: Permite enviar el código a un repositorio remoto
  \item \textbf{git pull}: Permite obtener el código desde un repositorio remoto
\end{itemize}


\subsubsection{Creando nuestro primer Fork}
\subsubsection{Creando un Pull-Request}
\subsubsection{Sistemas de integración continua (CI)}

\subsection{Introducción a la programación}

\subsubsection{Conceptos básicos de programación}
\subsubsection{Nuestro primer "Hola Mundo"}
\subsubsection{Guías de estilo}
\subsubsection{Introducción al TDD}
\subsubsection{Algunos ejercicios en diferentes lenguajes}

En el repositorio \url{https://github.com/erseco/ugr_tfm_maes_sample_exercises} hemos definido una serie de ejercicios así como sus correcciones, pasamos a detallar los mismos.


Escribe una función que encuentre el enésimo número primo. Para simplificar, asumimos que la entrada siempre será un entero (int). Si la entrada es menor o igual a 0, la función debe devolver -1.

pasos a seguir para crear un ejercicio

tiempos de creación de ejecución por el alumno

Dichos ejercicios se han incorporado en la sección de Anexos