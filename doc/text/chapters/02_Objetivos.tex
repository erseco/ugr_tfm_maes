\chapter{Objetivos}

El objetivo máximo de este proyecto es definir una metodología de trabajo para enseñar programación haciendo uso de sistemas de control de versiones de código que además, mediante un sistema software, permita la autocorrección de dichos ejercicios.

\bigskip
Dicho objetivo se descompone los siguientes objetivos principales:

\begin{itemize}
  \item \textbf{OBJ-1.} Analizar las posibilidades de los sistemas de autocorrección actualmente existentes.
  \item \textbf{OBJ-2.} Analizar el estado del arte actual en cuanto a herramientas de autocorrección.
  \item \textbf{OBJ-3.} Determinar las ventaja que nos brindan los sistemas de autocorrección para mejorar la enseñanza y aprendizaje de la programación.
  \item \textbf{OBJ-4.} Definir una metodología de trabajo para la enseñanza de la programación haciendo uso de sistemas de control de versiones.

\end{itemize}

Además como objetivos secundarios tendremos:

\begin{itemize}
  \item \textbf{OBJ-5.} Desarrollar una herramienta para la autocorrección de ejercicios de programación.
  \item \textbf{OBJ-6.} Escribir una serie de ejercicios de ejemplo para comprobar el correcto funcionamiento de la herramienta.
  \item \textbf{OBJ-7.} Definir las posibilidades adicionales de la herramienta tanto para la enseñanza como para el aprendizaje de la programación.
\end{itemize}

\section{Alcance de los objetivos}
El fin inmediato de este informe es definir una metodología de trabajo para enseñar programación. Como objetivo adicional intentaremos desarrollar una herramienta que permita a los profesores definir una serie de ejercicios que los propios alumnos puedan corregir de forma autónoma y automática de una forma sencilla.\\

Además todo el código así como la documentación resultante se liberará con una licencia libre para que cualquiera pueda hacer uso de la metodología, del código desarrollado así como de las conclusiones obtenidas.

\section{Interdependencia de los objetivos}

Todos los objetivos son interdependientes entre sí, pero el cuarto objetivo (\textbf{OBJ-4}) es el principal motivador de este proyecto, por lo que aún sin representar el desarrollo de ningún trabajo en concreto es el que va a escudar y avalar el desarrollo de los demás. En aspectos más relacionados con la realización del proyecto, el tercer objetivo (\textbf{OBJ-3}) es el que nos brindará la base pedagógica sobre la que trabajar, ya que sienta la base sobre la que aplicar dicho cuarto objetivo (\textbf{OBJ-4}). El resto de objetivos secundarios, al no tener un carácter urgente serán resueltos en base a la disponibilidad del tiempo necesario para su realización.

\section{Conocimientos y herramientas utilizadas}

Destacar en los aspectos formativos previos más utilizados para el desarrollo de esta metodología los conocimientos adquiridos en la asignatura ``Procesos y contextos educativos'' en todo lo referente a legislación y metodologías de enseñanza, la asignaturas ``Innovación docente e Investigación Educativa en Ciencia y Tecnología'' por las ideas sobre como realizar una propuesta innovadora siendo ambas del \textbf{\master}.\\

También me gustaría destacar los conocimientos obtenidos en ``Cloud Computing'' para el análisis y configuración de los sistemas de integración continua y ``Planificación y Gestión de. Proyectos Informáticos.'' para definir los requisitos y el planteamiento inicial del proyecto, ambas del \textbf{Máster Profesional en Ingeniería Informática} así como las asignaturas ``Fundamentos de Programación'' y ``Programación Orientada a Objetos'' del \textbf{Grado en Ingeniería Informática} para el desarrollo de los ejercicios de ejemplo.\\

\bigskip
Para la realización de cada una de las partes se han usado multitud de herramientas específicas tales como pueden ser \texttt{LaTeX}, \texttt{Zotero}, \texttt{Travis CI} \texttt{GitHub} y \texttt{Git} entre otras.
