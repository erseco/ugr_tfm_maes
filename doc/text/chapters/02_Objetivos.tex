\chapter{Objetivos}

El objetivo máximo de este proyecto es la elaboración de un sistema software que permita la autocorrección de ejercicios de programación.

\bigskip
Dicho objetivo se descompone los siguientes objetivos principales:

\begin{itemize}
  \item \textbf{OBJ-1.} Desarrollar una herramienta para la autocorrección de ejercicios de programación.
  \item \textbf{OBJ-2.} Analizar el estado del arte actual en cuanto a herramientas de autocorrección.
  \item \textbf{OBJ-3.} Analizar las posibilidades de los sistemas de autocorrección actualmente analizados.
  \item \textbf{OBJ-4.} Determinar la ventaja que nos brindan los sistemas de autocorrección para mejorar la ensañanza y aprendizaje de la programación.
\end{itemize}

Además como objetivos secundarios tendremos:

\begin{itemize}
  \item \textbf{OBJ-5.} Desarrollar una serie de ejercicios de ejemplo para comprobar el correcto funcionamiento de la herramienta.
  \item \textbf{OBJ-6.} Definir los usos posibles de la herramienta tanto para la enseñanza como para el aprendizaje de la programación.
\end{itemize}


\section{Alcance de los objetivos}
El fin inmediato de este informe es desarrollar una herramienta que permita a los profesores definir una serie de ejercicios que los propios alumnos puedan corregir de forma autonoma y automática de una forma sencilla.
Además todo el código así como la documentación resultante se liberará con una licencia libre para que cualquiera pueda hacer de la herramienta desarrollada así como de las conclusiones obtenidas.

\section{Interdependencia de los objetivos}

Todos los objetivos son interdependientes entre sí, pero el primer objetivo (\textbf{OBJ-1}) es el principal motivador de este proyecto, por lo que aún sin representar el desarrollo de ningún trabajo en concreto es el que va a escudar y avalar el desarrollo de los demás. En aspectos más relacionados con la realización del proyecto, el tercer objetivo (\textbf{OBJ-3}) es el que nos brindará el sistema sobre la que trabajar, ya que sienta la base sobre la que aplicar el cuarto objetivo (\textbf{OBJ-4}). El resto de objetivos secundarios, al no tener un carácter urgente serán resueltos en base a la disponibilidad del tiempo necesario para su realización.

\section{Conocimientos y herramientas utilizadas}

\bigskip
Destacar en los aspectos formativos previos más utilizados para el desarrollo del proyecto los conocimientos adquiridos en las asignaturas ``Cloud Computing'' para el análisis y configuración de los diferentes servicios de red así como todo lo referente a virtualización de sistemas, ``Programación Orientada a Objetos'' para el desarrollo de los ejercicios de ejemplo, ``Planificación y Gestión de. Proyectos Informáticos.'' para definir los requisitos y el planteamiento inicial del proyecto, ``Seguridad en Sistemas Operativos'' para lo referente a las buenas prácticas de programación y ``Servidores Web de Altas Prestaciones'' para la realización de pruebas desde el punto de vista de disponibilidad y carga de trabajo.

Para la realización de cada una de las partes se han usado multitud de herramientas específicas tales como pueden ser \texttt{Docker}, \texttt{Travis} \texttt{GitHub}, \texttt{LaTeX} y \texttt{git} entre otras.
