\chapter{Objetivos}

El objetivo de este proyecto es definir una metodología de trabajo para enseñar programación haciendo uso de sistemas de control de versiones de código fuente que además, mediante un sistema software, permita la autocorrección de dichos ejercicios.



% Veo mezcla de objetivos con tareas para conseguirlos. Por ejemplo, comprobar que el sistema funciona correctamente es un objetivo, el uso de ejercicios de ejemplo es una tarea para alcanzarlo.

% Yo no diría que los ejercicos son únicamente para probar el funcionamiento del sistema, sino también para ilustrar cómo se emplearía (como guía para los profesores que quieran adoptarlo)

% \begin{itemize}
%   \item Reducir la tediosa tarea de la corrección de ejercicios.
%   \item Motivar la enseñanza y aprendizaje de la programación a través de ejemplos.
%   \item Ver las ventajas de usar sistemas de control de versiones.
%   \item Enseñar a nuestros alumnos como utilizar un sistema de control de versiones de forma correcta.
%   \item Mostrar nuevas prácticas de programación como puede ser la integración continua.
% \end{itemize}


\bigskip
Dicho objetivo se descompone en los siguientes subobjetivos principales:

\begin{itemize}
  \item \textbf{OBJ-1.} Analizar las posibilidades y el estado del arte actual de los sistemas de autocorrección actualmente existentes.
  \item \textbf{OBJ-2.} Determinar las ventaja que nos brindan los sistemas de autocorrección para mejorar la enseñanza y aprendizaje de la programación.
  \item \textbf{OBJ-3.} Definir una metodología de trabajo para la enseñanza de la programación haciendo uso de sistemas de control de versiones.

\end{itemize}

Además como objetivos secundarios tendremos:

\begin{itemize}
  \item \textbf{OBJ-4.} Desarrollar una herramienta para la autocorrección de ejercicios de programación.
  \item \textbf{OBJ-5.} Escribir una serie de ejercicios de ejemplo que sirvan como guía de cómo utilizar nuestra metodología.
  \item \textbf{OBJ-6.} Definir las posibilidades adicionales de la metodología tanto para la enseñanza como para el aprendizaje de la programación.
\end{itemize}

\section{Alcance de los objetivos}
El fin inmediato de este informe es definir una metodología de trabajo para enseñar programación. Como objetivo adicional intentaremos desarrollar una herramienta que permita a los profesores definir una serie de ejercicios que los propios alumnos puedan corregir de forma autónoma y automática de una manera sencilla.

\bigskip
Además todo el código así como la documentación resultante se ha liberado con una licencia libre para que cualquiera pueda hacer uso de la metodología, del código desarrollado así como de las conclusiones obtenidas.

\section{Interdependencia de los objetivos}

Todos los objetivos son interdependientes entre sí, pero el cuarto objetivo (\textbf{OBJ-3}) ha sido el principal motivador de este proyecto y es el que ha escudado y avalado el desarrollo de los demás. En aspectos más relacionados con la realización del proyecto, el tercer objetivo (\textbf{OBJ-3}) es el que nos ha brindado la base pedagógica sobre la que trabajar, ya que ha sentado la base sobre la que aplicar dicho cuarto objetivo (\textbf{OBJ-3}). El resto de objetivos secundarios, al no tener un carácter urgente han resueltos en base a la disponibilidad del tiempo necesario para su realización.

\section{Conocimientos y herramientas utilizadas}

Destacar en los aspectos formativos previos más utilizados para el desarrollo de esta metodología los conocimientos adquiridos en la asignatura \texttt{Procesos y contextos educativos} en todo lo referente a legislación y metodologías de enseñanza y la asignatura \texttt{Innovación docente e Investigación Educativa} por las ideas sobre cómo realizar una propuesta innovadora siendo ambas del \textbf{\master}.

\bigskip
También me gustaría destacar los conocimientos obtenidos en \texttt{Cloud Computing} para el análisis y configuración de los sistemas de integración continua y \texttt{Planificación y Gestión de Proyectos Informáticos.} para definir los requisitos y el planteamiento inicial del proyecto, ambas del \textbf{Máster Profesional en Ingeniería Informática} así como las asignaturas \texttt{Fundamentos de Programación} y \texttt{Programación Orientada a Objetos} del \textbf{Grado en Ingeniería Informática} para el desarrollo de los ejercicios de ejemplo.

\bigskip
Para la realización de cada una de las partes se han usado multitud de herramientas específicas tales como \texttt{LaTeX}, \texttt{Zotero}, \texttt{Travis CI} \texttt{Git} y \texttt{GitHub} entre otras.
