\chapter{Propuesta Metodológica}

En este capítulo vamos a definir y a detallar los procesos que vamos a seguir para llevar a cabo la propuesta pedagógica definida en el capítulo anterior.

\section{Herramientas necesarias}

\subsection{Sistemas de control de versiones}

Lo primero que necesitaremos es enseñar a nuestros alumnos a utilizar un sistema de control de versiones, aunque existen varios sistemas (SVN, Mercurial, CVS), hoy en día estándar de facto es Git que además es software libre.

Para una administración más sencilla de nuestro código utilizaremos un cliente web como es GitHub, aunque existen otros como pueden ser GitLab o BitBucket.

\subsubsection {Git}

Git es un sistema de control de versiones distribuido que nos ayuda a llevar un seguimiento de los cambios en el código mientras desarrollamos. Se diseñó para su uso con código fuente pero se puede utilizar para llevar el seguimiento de los cambios en cualquier conjunto de archivos. Sus características incluyen la velocidad, la integridad de los datos y la compatibilidad con flujos de trabajo distribuidos y no lineales.

Git fue creado por Linus Torvalds en 2005 para el desarrollo del núcleo de Linux como alternativa al software BitKeeper que era un sistema de control de versiones propietario tras cambios en la licencia de este último.

Como con la mayoría de los sistemas de control de versiones distribuidos, y a diferencia de la mayoría de los sistemas cliente-servidor cada directorio de Git es un repositorio completo con un historial completo y capacidades de seguimiento de versiones completas, independiente del acceso a la red o a un servidor central.

Git es software libre y de código abierto distribuido bajo los términos de la Licencia Pública General GNU versión 2.

\subsubsection {GitHub}

GitHub es un servicio de alojamiento basado en web para el control de versiones a través de Git, fue adquirido por MicroSoft en 2018. Se utiliza sobre todo para código fuente aunque se puede utilizar para almacenar todo tipo de contenidos, incluso libros. Ofrece toda la funcionalidad de control de versiones distribuido y gestión de código fuente de Git, además de añadir sus propias características.

Proporciona control de acceso y varias funciones de colaboración como seguimiento de errores, \textit{pull requests}, gestión de tareas y \textit{wikis} para cada proyecto. Además cuenta con un sub-proyecto llamado GitHub Education que es ampliamente utilizado como herramientas para la formación de programadores (\cite{hernandez_integracion_2018}).


\section {Sistemas de integración continua (CI)}

\subsubsection {Travis-CI}

Travis CI es un servicio alojado de integración continua que se utiliza para construir y testear proyectos de software alojados en GitHub.

Los proyectos de código abierto pueden ser probados de forma gratuita a través del sitio \url{https://travis-ci.org}. En cambio los proyectos privados deben pagar para poder ser ejecutados a través del sitio \url{https://travis-ci.com}.

Gran parte de su código fuente es software libre y está disponible en GitHub.

\subsection{Lenguajes utilizados}

Al igual que no existe una única solución a un problema, existen multitud de lenguajes de programación, nosotros nos vamos a centrar en solamente cuatro de ellos ya que son los que cumplen con los contenidos de la mayoría de las asignaturas sobre informática, dichos lenguajes son Python, Ruby, C y HTML, aun así esta metodología es aplicable a otros lenguajes por lo que se podría utilizar para enseñar Java, Go, Haskell o Javascript:

\subsubsection{Python}

Python es un lenguaje de programación desarrollado por Guido van Rossum en 1991 cuya sintaxis favorece escribir código muy legible. Hoy en día es uno de los lenguajes de programación más utilizados en cursos de introducción a la programación ya que su sintaxis suele ser más sencilla, asemejándose al ``pseudocódigo''.

Según diversos índices (\cite{TIOBE2019}) y encuestas (\cite{stack_overflow_stack_2019}) Python es uno de los lenguajes con mayor proyección de futuro y que del que más se ha incrementado su uso, sobre todo motivado por su sencillez a la hora de aprenderlo y a su potencia como lenguaje de análisis de datos habiendo prácticamente desbancando al lenguaje estadístico R.

\bigskip
Vamos a centrar nuestros ejercicios en la versión 3 de Python ya que la versión 2 actualmente solo recibe actualizaciones de seguridad y en enero de 2020 dejará de estar oficialmente soportada (\cite{python.org_pep_2018}).

\subsubsection{Ruby}

Ruby es un lenguaje de programación orientado a objetos creado por el programador japonés Yukihiro Matsumoto en 1995. Su orientación a objetos es denominada ``fuerte'' ya que todos los tipos de dato son a su vez un objeto.

\bigskip
Ruby empezó a ganar popularidad tras la publicación del framework de aplicaciones web Ruby On Rails (RoR) por parte de David Heinemeier Hansson en 2005 ya que el mismo simplificaba muchísimo el desarrollo siguiendo el patrón Modelo Vista Controlador (MVC).

\bigskip
Debido a su sencilla sintaxis y su fuerte orientación a objetos es ampliamente utilizado para enseñar las particularidades de dicho paradigma.

\subsubsection{C}

El lenguaje de programación C es un lenguaje de propósito general desarrollado por Dennis Ritchie en los Laboratorios Bell entre 1969 y 1972, es el lenguaje de programación más popular para crear software para sistemas embebidos y micro-controladores, aunque también se puede utilizar para crear aplicaciones.

\bigskip
Es un lenguaje tipado estáticamente (en tiempo de compilación) y considerad de de medio nivel ya que dispone de las estructuras típicas de los lenguajes de alto nivel pero permitiendo un control a muy bajo nivel. Los compiladores suelen ofrecer extensiones al lenguaje que posibilitan mezclar código en ensamblador con código C, acceder a memoria o controlar diferentes dispositivos.

\subsubsection{HTML}

El lenguaje de marcas de hipertexto o HTML por sus siglas en inglés, es un lenguaje de marcado para la elaboración de páginas web. La primera versión del lenguaje fue creada por Tim Berners-Lee en 1991 mientras trabajaba en el Centro Europeo de Investigaciones Nucleares (CERN) en Suiza. Actualmente sus especificaciones están a cargo del World Wide Web Consortium (W3C).

\bigskip
HTML no es un lenguaje de programación, es un lenguaje de marcado que sirve para definir documentos estandarizados.

\section {Herramientas de análisis de código}

También conocidos como ``linters''  son herramientas que analizan el código fuente para marcar errores de programación, \textit{bugs}, errores estilísticos y código sospechosos. Algunos son capaces incluso de calcular la complejidad ciclomática\footnote{Valor escalar que mide la complejidad lógica de un programa.} de un algoritmo. El término proviene de una antigua utilidad de Unix para análisis de código fuente en lenguaje C llamada ``lint''.

\bigskip
Existen \textit{linters} para prácticamente todos los lenguajes de programación, en nuestros ejemplos hemos utilizado los siguientes:

\subsection {Pycodestyle}

Pycodestyle es una herramienta para comprobar código fuente en el lenguaje Python contra las convenciones de estilo PEP8.

\bigskip
PEP8 es una guía de estilo de codificación para Python definida inicialmente por Guido van Rossum, creador del lenguaje Python y que está considerada como la estándar del lenguaje.

\subsection{Rubocop}

Es un analizar de código para el lenguaje Ruby que tiene opciones muy interesantes como el cálculo de la complejidad ciclomática y la auto-reparación de código fuente incorrecto.

\subsection{Cpplint}

Herramienta de código abierto desarrollada por Google, diseñada para garantizar que el código C y C++ se ajusta a las guías de estilo de codificación de la compañía.

\subsection {HTML Tidy}

Es una aplicación de consola que sirve para corregir código HTML no válido, detectar posibles errores de accesibilidad web y mejorar el diseño y el estilo de sangría del marcado resultante. Fue desarrollado en 2002 por el miembro del World Wide Web Consortium (W3C) Dave Raggett.

\section {Herramientas de verificación de código}

Las herramientas de verificación de código se utilizan para ejecutar los test TDD o BDD en nuestro código fuente. En nuestros ejercicios vamos a utilizar las siguientes  :

\subsection {Pytest}

Pytest es un framework de test unitarios para el lenguaje de programación Python que facilita la creación de pruebas sencillas y escalables. Las pruebas son expresivas y legibles y no se requiere código adicional.

\subsection {RSpec}

RSpec es una herramienta para realizar test BDD que sirve para probar código escrito en el lenguaje de programación Ruby.

\subsection{MinUnit}

MinUnit es un mini-framework para correr test unitarios en lenguaje C. Su código fuente se puede encontrar en \url{http://www.jera.com/techinfo/jtns/jtn002.html}. Es particularmente pequeño ya que consiste en un fichero de cabecera ``.h'' de tan solo 4 líneas de código.

\section{Metodología de aprendizaje}

Algunas de las herramientas necesarias para nuestro sistema de corrección requieren algo de formación previa, por lo que vamos a definir una metodología de aprendizaje para las mismas

\subsection{Aprendiendo a usar Git}

Como ya explicamos anteriormente Git es uno de los sistemas de control de versiones más utilizados en la actualidad. Es difícil encontrar hoy en día una empresa que se dedique a la programación de forma profesional que no la utilice. A pesar de ello no forma parte del currículo de los ciclos formativos de informática. De hecho, salvo contadas excepciones, tampoco se aprende su uso en los Grados en Ingeniería Informática. De ahí la ventaja de nuestro sistema, ya que vamos a aprovechar para enseñarles a utilizar una herramienta que van a usar de forma exhaustiva durante su futura carrera profesional.

\bigskip
Para aprender a utilizar Git nos vamos a basar en diversos manuales que enseñan a utilizarlo de una forma sencilla (\cite{popov_control_2012}) sin necesidad de tener conocimientos de programación. Como ya indicamos en la propuesta pedagógica vamos a definir unos contenidos comunes que son los siguientes:

\subsubsection{Introducción a Git}

En esta parte les explicaremos que es un sistema de control de versiones, que diferencias hay entre un sistema centralizado y uno distribuido. También les hablaremos de la motivación de por qué usarlos poniéndoles ejemplos de lo que seguramente ellos han utilizado hasta ahora que seguramente será duplicar los archivos en distintas carpetas. También les hablaremos un poco de la historia de Git.

\subsubsection{Creando una cuenta en GitHub}

Aquí les indicaremos como crear su primera cuenta en GitHub, para ello deberán acceder a la página web \url{https://github.com} y hacer clic en SignUp (figura \ref{fig:git1}).

\begin{figure}[H]
\centering
\includegraphics[width=1.0\textwidth]{../images/git1}
\caption{Pagina de registro de GitHub}
\label{fig:git1}
\end{figure}

\subsubsection{Creando nuestro primer repositorio}

Crear un repositorio en GitHub es muy intuitivo, solo tenemos que hacer clic en el botón New y rellenar los datos que nos solicita (figura \ref{fig:git2}). De igual forma podemos crear un repositorio de forma local en nuestro ordenador ejecutando el comando \texttt{git init}

\begin{figure}[H]
\centering
\includegraphics[width=1.0\textwidth]{../images/git2}
\caption{Ventana de creación de repositorio en GitHub}
\label{fig:git2}
\end{figure}

Para que los alumnos se familiaricen con el uso de GitHub editaremos el fichero \textcc{README.md} con el editor WYSIWYG\footnote{acrónimo de What You See Is What You Get (en español, "lo que ves es lo que obtienes").} integrado en la plataforma.

\subsubsection{Comandos básicos de Git}

Aquí podemos ver algunos de los comandos más habituales de Git, se puede encontrar la referencia completa en su documentación oficial en \url{https://git-scm.com/doc}.

\begin{itemize}
  \item \textbf{git clone}: Clona un repositorio
  \item \textbf{git status}: Nos dice el estado de un repositorio
  \item \textbf{git commit}: Nos permite guardar los cambios en una rama
  \item \textbf{git checkout}: Nos permite cambiar de rama
  \item \textbf{git branch}: Nos permite crear y listar ramas
  \item \textbf{git push}: Permite enviar el código a un repositorio remoto
  \item \textbf{git pull}: Permite obtener el código desde un repositorio remoto
\end{itemize}

Para que los alumnos se familiaricen con el uso de estos comandos vamos a pedirle que hagan cambios en los archivos de la carpeta \textcc{homework_00_markdown} con su editor y hagan un \textit{commit} y un \textit{push} con el código fuente.

\subsubsection{Creando nuestro primer Fork}

Una bifurcación, o fork en inglés, es el término que se utiliza para indicar una ramificación de un trabajo. Básicamente significa que vamos a copiar un proyecto y crear uno nuevo haciéndole modificaciones. La capacidad de crear bifurcaciones de código de forma sencilla es una de las características que han ayudado a la plataforma GitHub llegar a ser el sitio de referencia para albergar proyectos de software libre.

\bigskip
Para crear un fork en GitHub de cualquier proyecto simplemente tenemos que hacer click en el botón situado a la derecha de cada proyecto (figura \ref{fig:git_fork}). Una cosa a tener en cuenta a la hora de hacer un Fork de un proyecto es la licencia bajo la que esté dicho código, el cual suele venir indicado en el fichero LICENSE, no todas las licencias permiten la libre distribución de proyectos derivados.

\begin{figure}[H]
\centering
\includegraphics[width=1.0\textwidth]{../images/git_fork}
\caption{Detalle del botón de Fork en GitHub}
\label{fig:git_fork}
\end{figure}

\subsubsection{Creando un Pull-Request}

Las contribuciones a un repositorio de código fuente que utiliza un sistema de control de versiones distribuido se realizan comúnmente por medio de un ``pull request''. El colaborador solicita que el encargado del proyecto haga un ``pull'' con los cambios en el código fuente, de ahí el nombre. El mantenedor puede revisar el conjunto de cambios, discutir modificaciones potenciales o mezclar el código.

Dependiendo del flujo de trabajo establecido el código puede ser probado antes de ser incluido en la versión oficial. Algunos proyectos ejecutan un conjunto de pruebas automatizadas en cada solicitud de extracción, utilizando una herramienta de integración continua como Travis CI, y el revisor verifica que cualquier código nuevo tenga la cobertura de pruebas adecuada.

Para hacer un ``Pull request'' haremos clic en el botón ``New pull request'' y seleccionando que ramas queremos fusionar (figura \ref{fig:git_pr}).

\begin{figure}[H]
\centering
\includegraphics[width=1.0\textwidth]{../images/git_pr}
\caption{Detalle de creación de un Pull Request en GitHub}
\label{fig:git_pr}
\end{figure}

\subsubsection{Sistemas de integración continua (CI)}

Un sistema de integración continua suele consistir en una plataforma que ejecuta una serie de pasos con cada ``Push'' que realizamos a nuestro sistema de control de versiones. A esta serie de pasos se le suele denominar ``pipeline'' y suele contener pasos habituales como pueden ser la compilación, la ejecución de validaciones sintácticas y estilísticas de código (\textit{linter}) y la ejecución de los diferentes test para comprobar que efectivamente el código funciona.

\bigskip
Como ya hemos indicado, para nuestra metodología vamos a utilizar ``Travis CI'' aunque la forma de funcionar es muy similar en casi todas las plataformas. Para crear una cuenta en ``Travis CI'' iremos a la url \url{https://travis-ci.org} y haremos click en el botón ``Sign-Up'' (figura \ref{fig:travis_signup})).

\begin{figure}[H]
\centering
\includegraphics[width=1.0\textwidth]{../images/travis_signup}
\caption{Detalle de creación de una cuenta en Travis CI}
\label{fig:travis_signup}
\end{figure}

Una vez tengamos nuestra cuenta tenemos que activar en cuales de nuestra lista de repositorios queremos activar el servicio. Para ellos solo debemos hacer click en el interruptor hasta que quede de color verde (figura \ref{fig:travis_enable}).

\begin{figure}[H]
\centering
\includegraphics[width=1.0\textwidth]{../images/travis_enable}
\caption{Detalle de activación de un repositorio en Travis CI}
\label{fig:travis_enable}
\end{figure}

Una vez activado, con cada \texttt{git push} se ejecutará el ``pipeline'' definido en el fichero \texttt{.travis.yml}, en nuestro repositorio de ejemplo se puede obtener uno configurado para ejecutar los diferentes ejemplos realizados para mostrar el funcionamiento de la metodología.

\subsection{Introducción a la programación}

Una vez los alumnos han aprendido los comandos básicos de Git y se han dado de alta en GitHub es hora de enseñarles algunos conceptos básicos de programación.

\subsubsection{Conceptos básicos de programación}

Todos los lenguajes de programación comparten algunos elementos básicos que funcionan y se usan de forma diferente en cada lenguaje, pero que cumplen el mismo objetivo. Esos elementos son:

\begin{itemize}
  \item Tipos de datos: Enteros, Decimales, Caracteres, Cadenas de texto, etc...
  \item Variables: donde almacenar los datos.
  \item Control de flujo: los ``if''
  \item Bucles: los conocidos ``loops'', ``for'' y ``while''
  \item Funciones
  \item Entra y Salida: pintando en pantalla.
\end{itemize}

Hay que remarcar que estamos explicando los conceptos básicos de programación, por eso esto no incluye ni estructuras de datos, ni orientación a objetos ni recursividad, realmente eso lo aprenderán en ejercicios adicionales.

\subsubsection{Nuestro primer ``Hola Mundo''}

Un ``Hola mundo'' es un programa cuya única finalidad es escribir la frase ``Hola mundo!''. Este programa se usa como introducción en la mayoría de lenguajes de programación siendo el primer ejercicio típico, y se considera fundamental desde el punto de vista didáctico. Una implementación de dicho programa se puede encontrar para prácticamente todos los lenguajes de programación existentes.

\bigskip
En nuestra metodología animamos a los profesores a enseñar un primer ejemplo del lenguaje a impartir usando un ``Hola Mundo'' para que los alumnos se familiaricen con el lenguaje. En nuestros ejercicios de ejemplo vamos a incorporar el hola mundo como primer ejercicio de programación.

\subsubsection{Guías de estilo}

En esta sección les enseñaremos la guía de estilo que vamos a seguir. Aunque es algo que se suele obviar en cursos de programación, a la hora de escribir código de calidad es importante seguir una guía de estilo, además de crear buenas prácticas de programación. Les explicaremos lo importante que es comentar correctamente y se utilizará a poder ser una guía de estilo estandarizada para el lenguaje que vayamos a impartir.

\bigskip
Si el lenguaje que vamos a impartir no tuviera guía de estilo estandarizada, o si el profesor lo estima didáctico se podría consensuar con los alumnos el tipo de estilo que se va a seguir. Este caso es aplicable al uso de comillas simples '' o dobles "" pues no hay un consenso sobre cuales se deben usar, igual pasa con la indentación con espacios o con tabuladores.

\subsubsection{Introducción al TDD}

Aquí les explicaremos en consiste el TDD, que como ya hemos visto es escribir la prueba, escribir el código y una vez funcione refactorizar. En nuestro caso concreto les vamos a dar nosotros realizadas las pruebas TDD, pero les podremos animar a que realicen pruebas adicionales.

\subsubsection{Ejercicios de ejemplo}

En el repositorio \url{https://github.com/erseco/ugr_tfm_maes_sample_exercises} hemos definido una serie de ejercicios así como sus correcciones, dichos ejercicios se han incorporado en la sección de Anexos.

\subsection{Usando la integración continua para aprender}

Una vez hemos visto como usar Git y los conceptos básicos de programación vamos a ver algunos de los errores con los que se pueden encontrar los alumnos, como pueden usar el sistema de integración continua para detectarlos ellos mismos y como corregirlos (figura \ref{fig:linter_error_python}).

\begin{figure}[H]
\centering
\includegraphics[width=1.0\textwidth]{../images/linter_error_python}
\caption{Detalle de un error de linter python en un pipeline}
\label{fig:linter_error_python}
\end{figure}


