\chapter{Conclusiones}

Tras el análisis de las numerosas herramientas de corrección y verificación existentes, así como la retroalimentación recibida parte de profesores y alumnos, sin olvidar a las personas que respondieron a la encuesta realizada he concluido que, a pesar de tener algunos inconvenientes, los sistema de autocorrección nos proveen de una serie de ventajas muy a tener en cuenta.

Aun así, dichos sistemas tienen algunos inconvenientes. \textit{Coderunner}, por ejemplo, depende de que tengamos una instalación de Moodle, y \textit{Pythontutor} está limitado al aprendizaje del lenguaje Python.

Además, dichos sistemas existentes, necesitan de un aprendizaje previo.

Nuestro sistema tiene la ventaja de utilizar tecnologías estándares con amplia difusión en la industria informática con lo que no van a aprender a usar una herramienta que no usarán jamás cuando finalicen la asignatura, todo lo contrario, les vamos a enseñar a usar tecnologías con las que se van a tener que enfrentar más tarde o más temprano.

En mis años de Universidad jamás me hablaron de los \textit{linters}, de las guías de estilo de codificación y de las pruebas unitarias (TDD) se vio de pasada. Tampoco aparece el uso de estas herramientas en ninguna de las asignaturas relacionadas con la informática de Educación Secundaria, Bachillerato y Formación Profesional. Hay una asignatura llamada ``Entornos de desarrollo'' que enseña a usar algunos IDEs\footnote{Entorno de Desarrollo Integrado, Integrated Development Environment en inglés.} no contempla ninguna de las tecnologías mencionadas.

Esta metodología sirve para enseñar cualquier lenguaje de programación o paradigma de programación existente, es decir, es completamente agnóstica del lenguaje.

Al guardarse un registro de todo lo que se envía al repositorio de código junto al resultado de su compilación, el profesor puede tener métricas muy detalladas de los errores cometidos por cada alumno, ver su evolución e incluso el tiempo dedicado.

Observando el \textit{pipeline} los profesores pueden ver de forma global donde están fallando los alumnos y realizar modificaciones sobre los ejercicios para incidir en los temas en los que los alumnos tengan mayor dificultad.

Además, al ser software libre y estar disponible de forma pública cualquiera puede hacer uso de los ejercicios e incluso proponer mejoras, a su vez los alumnos pueden enseñar su código en futuras entrevistas de trabajo. A día de hoy el mejor currículum vitae de un programador informático es su perfil en GitHub.

Como inconvenientes podemos destacar que gran parte de los docentes tampoco están habituados a usar los sistemas utilizados por lo que deberían formarse en su utilización. Además la curva de trabajo para el profesor es más pronunciada pues tiene que preparar todos los ejercicios  gran variedad de ejercicios y test antes del comienzo de las clases.

Otro inconveniente es que el profesor debe ir modificando regularmente los ejercicios ya que al estar disponibles de forma pública una simple búsqueda les permitiría a los alumnos obtener la resolución de los ejercicios de cursos anteriores.

Aun con estos inconvenientes vemos que las ventajas son mucho mayores, y sumándole la retroalimentación recibida parte de profesores y alumnos además de las personas que respondieron a la encuesta podemos concluir que nuestro sistema aporta un valor añadido evidente a los sistemas existentes de corrección y espero implantar la metodología en mis propias clases.
