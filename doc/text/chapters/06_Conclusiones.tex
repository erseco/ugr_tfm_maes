\chapter{Conclusiones}

Una vez realizada tanto la propuesta pedagógica como la propuesta metodológica podemos ver como a pesar de sus inconvenientes los sistema de autocorrección nos proveen de diversas ventajas.

\section{Ventajas}

Utiliza sistemas modernos

Al estar la corrección basada en los tests definidos por ele profesor la solución es agnóstica del lenguaje, el mismo método nos puede servir para enseñar cualquier lenguaje de programación o incluso un paradigma concreto.

Podemos usar el mismo método tanto para cursos de iniciación como para cursos mucho más complejos.

Como ya hemos visto en la propuesta pedagógica podemos definir las pautas pedagógicas de la corrección.

Tal y como dice (REFERENCIA ESTUDIO), mejora la interacción con el alumno.

Al guardarse un registro de todo lo que se envía al repositorio de código junto al resultado de su compilación, el profesor puede tener métricas muy detalladas de los errores cometidos por cada alumno, ver su evolución e incluso el tiempo dedicado.

Es fácil detectar el plagio.

El profesor puede ir haciendo futuras modificaciones sobre los ejercicios para incidir en los temas en los que los alumnos tengan mayor dificultad.


\section{Inconvenientes}

El profesor tiene que estar atento para evitar el plagio, asimismo debe de cambiar regularmente los ejercicios ya que al estar disponibles en Internet una simple búsqueda les permitiría a los alumnos obtener la resolución de los ejercicios de cursos anteriores.

La curva de trabajo para el profesor es mas pronunciada al principio pues debe de preparar una gran variedad de ejercicios y test antes del comienzo de las clases.



