\chapter{Anexos}

\section{Código fuente}

\subsection{Pipeline para Travis-CI}

\begin{lstlisting}[language=yaml,caption={.travis.yml},captionpos=b]
---
dist: xenial

before_install:
  - pyenv global system 3.6.7
  - sudo apt-get update
  - sudo apt-get install -y tidy cppcheck
install:
  - pip3 install pytest pycodestyle requests beautifulsoup4
  - gem install rspec rubocop mdl

script:
  - echo "Testing that we have all the requirements"
  - mdl --version
  - python3 --version
  - ruby --version
  - gcc --version
  - tidy --version

  - mdl homework_00_git/

  - pycodestyle homework_01_python/*.py
  - pytest --verbose homework_01_python/*.py

  - rubocop homework_02_ruby
  - rspec homework_02_ruby

  - cppcheck --verbose --error-exitcode=2 homework_03_c/*.c
  - make -C homework_03_c tests

  - tidy -quiet -errors homework_04_html/*.html
\end{lstlisting}

\subsection{Ejercicios Git/Markdown}
\begin{lstlisting}[language=markdown,caption={exercise\_01.md},captionpos=b]
# Hello world
\end{lstlisting}

\begin{lstlisting}[language=markdown,caption={exercise\_02.md},captionpos=b]
#h1 Heading 8-)

## h2 Heading

### h3 Heading

#### h4 Heading

##### h5 Heading

###### h6 Heading

## Horizontal Rules

___


## Typographic replacements

Enable typographer option to see result.

(c) (C) (r) (R) (tm) (TM) (p) (P) +-

test.. test... test..... test?..... test!....

!!!!!! ???? ,,  -- ---

"Smartypants, double quotes" and 'single quotes'

## Emphasis

Text **This is bold text**

Text __This is bold text__

Text *This is italic text*

Text _This is italic text_

Text ~~Strikethrough~~

## Blockquotes

> Blockquotes can also be nested...
>> ...by using additional greater-than signs right next to each other...
> > > ...or with spaces between arrows.

## Lists

Unordered

- Create a list by starting a line with `+`, `-`, or `*`
- Sub-lists are made by indenting 2 spaces:
  - Marker character change forces new list start:
- Very easy!

## Code

Block code "fences"

```
Sample text here...
```

## Tables

| Option | Description |
| ------ | ----------- |
| data   | path to data files to supply the data that will be passed. |
| engine | engine to be used for processing templates. |
| ext    | extension to be used for dest files. |

Right aligned columns

| Option | Description |
| ------:| -----------:|
| data   | path to data files to supply the data that will be passed. |
| engine | engine to be used for processing templates. |
| ext    | extension to be used for dest files. |

## Links

[link text](http://dev.nodeca.com)

[link with title](http://nodeca.github.io/pica/demo/ "title text!")

Autoconverted link https://github.com/nodeca/pica (enable linkify to see)
\end{lstlisting}

\subsection{Ejercicios Python}
\begin{lstlisting}[language=python,caption={exercise\_1.py},captionpos=b]
#!/usr/bin/env python3

# Question:
# Write a program that prints "Hello world!"
#
# Hints:
# Use the python print() built-infunction


print('Hola mundo!')
\end{lstlisting}

\begin{lstlisting}[language=python,caption={exercise\_2.py},captionpos=b]
#!/usr/bin/env python3

# Question:
# Write a function called add() which can compute and returns the summ of two
# given numbers.
#
# Suppose the following parameters are supplied to the program:
# 8 8
# Then, the output should be:
# 16

import sys


def add(x, y):
    return x + y


# ------- START TDD TESTS DEFINITION -----------
def test_add_8_8():
    assert add(8, 8) == 16


def test_add_4_5():
    assert add(4, 5) == 9
# ------- END TDD TESTS DEFINITION -------------


# Program entrypoint
if __name__ == "__main__":

    if len(sys.argv) == 3:

        number1 = int(sys.argv[1])
        number2 = int(sys.argv[2])
        result = add(number1, number2)
        print(result)

    else:
        print("Usage: %s <number1> <number2>" % sys.argv[0])
\end{lstlisting}

\begin{lstlisting}[language=python,caption={exercise\_3.py},captionpos=b]
#!/usr/bin/env python3

# Question:
# Write a function called get_factorial() which can compute and returns the
# factorial of a given number.
#
# Suppose the following parameter is supplied to the program:
# 8
# Then, the output should be:
# 40320
#
# Hints:
# You can't use the `math` module.

import sys


def get_factorial(number):
    if number == 0:
        return 1
    return number * get_factorial(number - 1)


# ------- START TDD TESTS DEFINITION -----------
def test_get_factorial_8():
    import math
    assert math.factorial(8) == get_factorial(8)


def test_get_factorial_3():
    import math
    assert math.factorial(3) == get_factorial(3)


def test_get_factorial_25():
    import math
    assert math.factorial(25) == get_factorial(25)
# ------- END TDD TESTS DEFINITION -------------


# Program entrypoint
if __name__ == "__main__":

    if len(sys.argv) == 2:
        number = int(sys.argv[1])
        result = get_factorial(number)
        print(result)

    else:
        print("Usage: %s <number>" % sys.argv[0])
\end{lstlisting}

\begin{lstlisting}[language=python,caption={exercise\_4.py},captionpos=b]
#!/usr/bin/env python3

# Question:
# Write a function that get the content of the <title> tag of a given url
#
# Suppose the following parameter is supplied to the program:
# https://www.google.es
# Then, the output should be:
# Google
#
# Hints:
# Use the requests module to read the web content
# Use the BeautifulSoup to process the HTML

import sys
import requests
from bs4 import BeautifulSoup


def get_title(url):

    r = requests.get(url)
    soup = BeautifulSoup(r.text, features="html.parser")

    return soup.title.contents[0]


# ------- START TDD TESTS DEFINITION -----------
def test_get_title_google():
    url = "https://www.google.com/"
    assert get_title(url) == "Google"


def test_get_title_alhambra():
    url = "http://www.alhambra-patronato.es/"
    assert get_title(url) == "Patronato de la Alhambra y Generalife"


def test_get_title_wikipedia():
    url = "https://en.wikipedia.org/wiki/Main_Page"
    assert get_title(url) == "Wikipedia, the free encyclopedia"
# ------- END TDD TESTS DEFINITION -------------


# Program entrypoint
if __name__ == "__main__":

    if len(sys.argv) == 2:
        url = sys.argv[1]
        result = get_title(url)
        print(result)

    else:
        print("Usage: %s <url>" % sys.argv[0])
\end{lstlisting}

\begin{lstlisting}[language=python,caption={exercise\_5.py},captionpos=b]
#!/usr/bin/env python3

# Question:
# Write a funciton that works like a "Rock-Paper-Scissors" game, remember the
# rules:
# - Rock beats scissors
# - Scissors beats paper
# - Paper beats rock
#
# The result should be:
# - It's a tie!
# - Rock wins!
# - Scissors win!
# - Paper wins!
# - Invalid input!
#
# Suppose the following parameter is supplied to the program:
# rock scissors
# Then, the output should be:
# Rock wins!


def play_game(option1, option2):

    if option1 == option2:
        return("It's a tie!")
    elif option1 == 'rock':
        if option2 == 'scissors':
            return("Rock wins!")
        else:
            return("Paper wins!")
    elif option1 == 'scissors':
        if option2 == 'paper':
            return("Scissors win!")
        else:
            return("Rock wins!")
    elif option1 == 'paper':
        if option2 == 'rock':
            return("Paper wins!")
        else:
            return("Scissors win!")
    else:
        return("Invalid input!")


# ------- START TDD TESTS DEFINITION -----------
def test_play_game_rock_scissors():
    assert play_game("rock", "scissors") == "Rock wins!"


def test_play_game_scissors_scissors():
    assert play_game("scissors", "scissors") == "It's a tie!"


def test_play_game_paper_rock():
    assert play_game("paper", "rock") == "Paper wins!"


def test_play_game_scissors_paper():
    assert play_game("scissors", "paper") == "Scissors win!"


def test_play_game_invalid_paper():
    assert play_game("invalid", "paper") == "Invalid input!"
# ------- END TDD TESTS DEFINITION -------------


# Program entrypoint
if __name__ == "__main__":

    if len(sys.argv) == 3:

        option1 = sys.argv[1]
        option2 = sys.argv[2]
        result = play_game(option1, option2)
        print(result)

    else:
        print("Usage: %s <option1> <option2>" % sys.argv[0])
\end{lstlisting}

\subsection{Ejercicios Ruby}
\begin{lstlisting}[language=ruby,caption={exercise\_1.rb},captionpos=b]
#!/usr/bin/env ruby
# frozen_string_literal: true

# Question:
# Write a program that prints "Hello world!"
#
# Hints:
# Use the ruby puts directive

puts 'Hello world!'
\end{lstlisting}

\begin{lstlisting}[language=ruby,caption={exercise\_2.rb},captionpos=b]
#!/usr/bin/env ruby
# frozen_string_literal: true

# Question:
# Write a function called add() which can compute and returns the summ of two
# given numbers.
#
# Suppose the following parameters are supplied to the program:
# 8 8
# Then, the output should be:
# 16

def add(number1, number2)
  number1 + number2
end

# Program entrypoint
if $PROGRAM_NAME == __FILE__
  if ARGV.length == 2

    number1 = ARGV[0].to_i
    number2 = ARGV[1].to_i
    result = add(number1, number2)
    puts result
  else
    puts "Usage: #{$PROGRAM_NAME} <number1> <number2>"
  end
end
\end{lstlisting}

\begin{lstlisting}[language=ruby,caption={exercise\_2\_spec.rb},captionpos=b]
# frozen_string_literal: true

require_relative '../exercise_2'

# ------- START TDD TESTS DEFINITION -----------
describe add(0, 0) do
  it 'returns 16 when passed 8 8' do
    expect(add(8, 8)).to eq 16
  end

  it 'returns 9 when passed 4 5' do
    expect(add(4, 5)).to eq 9
  end
end
# ------- END TDD TESTS DEFINITION -------------
\end{lstlisting}

\begin{lstlisting}[language=ruby,caption={exercise\_3.rb},captionpos=b]
#!/usr/bin/env ruby
# frozen_string_literal: true

# Question
# Write a function called get_factorial() which can compute and returns the
# factorial of a given number.
#
# Suppose the following parameter is supplied to the program
# 8
# Then, the output should be
# 40320
#
# Hints:
# You can't use the `math` module.

def get_factorial(number)
  return 1 if number.zero?

  number * get_factorial(number - 1)
end

# Program entrypoint
if $PROGRAM_NAME == __FILE__

  if ARGV.length == 1
    number = ARGV[0].to_i
    result = get_factorial(number)
    puts result
  else
    puts "Usage: #{$PROGRAM_NAME} <number>"
  end
end
\end{lstlisting}

\begin{lstlisting}[language=ruby,caption={exercise\_3\_spec.rb},captionpos=b]
# frozen_string_literal: true

require_relative '../exercise_3'

# ------- START TDD TESTS DEFINITION -----------
describe get_factorial(0) do
  it 'returns 40320 when passed 8' do
    expect(get_factorial(8)).to eq 40_320
  end

  it 'returns 6 when passed 3' do
    expect(get_factorial(3)).to eq 6
  end
  it 'returns 15511210043330985984000000 when passed 3' do
    expect(get_factorial(25)).to eq 15_511_210_043_330_985_984_000_000
  end
end
# ------- END TDD TESTS DEFINITION -------------
\end{lstlisting}

\begin{lstlisting}[language=ruby,caption={exercise\_4.rb},captionpos=b]
#!/usr/bin/env ruby

# Question:
# Write a function that get the content of the <title> tag of a given url
#
# Suppose the following parameter is supplied to the program:
# https://www.google.es
# Then, the output should be:
# Google
#
# Hints:
# Use the open-uri module to read the web content
# Use a regular expression to process the HTML
require "open-uri"

def get_title(url)
    open(url) do |f|
        str = f.read
        str.scan(/<title>(.*?)<\/title>/)
    end
end

# Program entrypoint
if __FILE__ == $PROGRAM_NAME

    if ARGV.length == 1
        url = ARGV[0]
        result = get_title(url)
        puts result

    else
        puts "Usage: #{$PROGRAM_NAME} <url>"
    end
end
\end{lstlisting}

\begin{lstlisting}[language=ruby,caption={exercise\_4\_spec.rb},captionpos=b]
# frozen_string_literal: true

require_relative '../exercise_4'

# ------- START TDD TESTS DEFINITION -----------
#  RSpec don't allow to connect to external websites, so we should mock the petition
describe get_factorial(0) do
    it 'returns "Google" when passed "https//www.google.com/"' do
      expect(get_title('https//www.google.com/')).to eq 'Google'
    end
    it 'returns "Patronato de la Alhambra y Generalife" when passed "http//www.alhambra-patronato.es/"' do
      expect(get_title('http//www.alhambra-patronato.es/')).to eq 'Patronato de la Alhambra y Generalife'
    end
    it 'returns "Wikipedia, the free encyclopedia" when passed "https//en.wikipedia.org/wiki/Main_Page"' do
      expect(get_title('https//en.wikipedia.org/wiki/Main_Page')).to eq 'Wikipedia, the free encyclopedia'
    end
end
# ------- END TDD TESTS DEFINITION -------------
\end{lstlisting}

\begin{lstlisting}[language=ruby,caption={exercise\_5.rb},captionpos=b]
#!/usr/bin/env ruby
# frozen_string_literal: true

# Question:
# Write a funciton that works like a "Rock-Paper-Scissors" game, remember the
# rules:
# - Rock beats scissors
# - Scissors beats paper
# - Paper beats rock
#
# The result should be:
# - It's a tie!
# - Rock wins!
# - Scissors win!
# - Paper wins!
# - Invalid input!
#
# Suppose the following parameter is supplied to the program:
# rock scissors
# Then, the output should be:
# Rock wins!

def play_game(option1, option2)
  if option1 == option2
    "It's a tie!"
  elsif option1 == 'rock'
    if option2 == 'scissors'
      'Rock wins!'
    else
      'Paper wins!'
    end
  elsif option1 == 'scissors'
    if option2 == 'paper'
      'Scissors win!'
    else
      'Rock wins!'
    end
  elsif option1 == 'paper'
    if option2 == 'rock'
      'Paper wins!'
    else
      'Scissors win!'
    end
  else
    'Invalid input!'
  end
end

# Program entrypoint
if $PROGRAM_NAME == __FILE__

  if ARGV.length == 2

    option1 = ARGV[0]
    option2 = ARGV[1]
    result = play_game(option1, option2)
    puts result

  else
    puts "Usage: #{$PROGRAM_NAME} <option1> <option2>"
  end
end
\end{lstlisting}

\begin{lstlisting}[language=ruby,caption={exercise\_5\_spec.rb},captionpos=b]
# frozen_string_literal: true

require_relative '../exercise_5'

# ------- START TDD TESTS DEFINITION -----------
describe play_game('', '') do
  it 'returns "Rock wins!" when passed "rock" "scissors"' do
    expect(play_game('rock', 'scissors')).to eq 'Rock wins!'
  end

  it 'returns "Its a tie!" when passed "scissors" "scissors"' do
    expect(play_game('scissors', 'scissors')).to eq "It's a tie!"
  end

  it 'returns "Paper wins!" when passed "paper" "rock"' do
    expect(play_game('paper', 'rock')).to eq 'Paper wins!'
  end

  it 'returns "Scissors win!" when passed ("scissors" "paper"' do
    expect(play_game('scissors', 'paper')).to eq 'Scissors win!'
  end

  it 'returns "Invalid input!" when passed "invalid" "paper"' do
    expect(play_game('invalid', 'paper')).to eq 'Invalid input!'
  end
end
# ------- END TDD TESTS DEFINITION -------------
\end{lstlisting}

\subsection{Ejercicios C}
\begin{lstlisting}[language=c,caption={Makefile},captionpos=b]
SRCS = $(wildcard *.c)
PROGS = $(patsubst %.c,%,$(SRCS))

all: $(PROGS)

%: %.c
  gcc $(CFLAGS) -o $@ $<

tests: test_2 test_3 test_5 clean

test_2:
  @gcc $(CFLAGS) -Wl, -e_main_tests -o exercise_2 exercise_2.c
  @./exercise_2

test_3:
  @gcc $(CFLAGS) -Wl, -e_main_tests -o exercise_3 exercise_3.c
  @./exercise_3

test_5:
  @gcc $(CFLAGS) -Wl, -e_main_tests -o exercise_5 exercise_5.c
  @./exercise_5

clean:
  @rm -f $(PROGS)
\end{lstlisting}

\begin{lstlisting}[language=c,caption={minunit.h},captionpos=b]
 /* file: minunit.h */
 #define mu_assert(message, test) do { if (!(test)) return message; } while (0)
 #define mu_run_test(test) do { char *message = test(); tests_run++; \
                                if (message) return message; } while (0)
int tests_run;

char * all_tests();

// This function acts as entrypoint for TDD
int main_tests(int argc, char **argv) {
     char *result = all_tests();
     if (result != 0) {
         printf("%s\n", result);
     }
     else {
         printf("ALL TESTS PASSED on %s\n", argv[0]);
     }
     printf("Tests run: %d\n", tests_run);

     return result != 0;
}
\end{lstlisting}

\begin{lstlisting}[language=c,caption={exercise\_1.c},captionpos=b]
/*
# Question:
# Write a program that prints "Hello world!"
#
# Hints:
# Use the c printf() built-infunction
*/

#include <stdio.h>

int main(void)
{
    printf("Hello world!\n");
    return 0;
}
\end{lstlisting}

\begin{lstlisting}[language=c,caption={exercise\_2.c},captionpos=b]
/*
# Question:
# Write a function called add() which can compute and returns the summ of two
# given numbers.
#
# Suppose the following parameters are supplied to the program:
# 8 8
# Then, the output should be:
# 16
*/

#include <stdio.h> //printf, scanf
#include <stdlib.h> //atoi, strtol
#include "minunit.h"

int add(int x, int y) {
    return x + y;
}

// ------- START TDD TESTS DEFINITION -----------
char * test_add_8_8(){
    mu_assert("error", add(8, 8) == 16);
    return 0;
}

char * test_add_4_5() {
    mu_assert("error", add(4, 5) == 9);
    return 0;
}

char * all_tests() {
     mu_run_test(test_add_8_8);
     mu_run_test(test_add_4_5);
     return 0;
 }
// ------- END TDD TESTS DEFINITION -------------

// Program entrypoint
int main(int argc, char **argv)
{

    if (argc == 3) {

        int number1 = atoi(argv[1]);
        int number2 = atoi(argv[2]);
        int result = add(number1, number2);
        printf("%d\n", result);

    } else {
        printf("Usage: %s <number1> <number2>\n", argv[0]);
    }

    return 0;
}
\end{lstlisting}

\begin{lstlisting}[language=c,caption={exercise\_3.c},captionpos=b]
/*
# Question:
# Write a function called get_factorial() which can compute and returns the
# factorial of a given number.
#
# Suppose the following parameter is supplied to the program:
# 8
# Then, the output should be:
# 40320
#
# Hints:
# You can't use the `math` module.
*/

#include <stdio.h> //printf, scanf
#include <stdlib.h> //atoi, strtol
#include "minunit.h"


int get_factorial(int number) {
    if (number == 0) {
        return 1;
    }
    return number * get_factorial(number - 1);
}

// ------- START TDD TESTS DEFINITION -----------
char * test_get_factorial_8(){
    mu_assert("error", get_factorial(8) == 40320);
    return 0;
}

char * test_get_factorial_3() {
    mu_assert("error", get_factorial(3) == 6);
    return 0;
}

char * test_get_factorial_10() {
    mu_assert("error", get_factorial(10) == 3628800);
    return 0;
}

char * all_tests() {
     mu_run_test(test_get_factorial_8);
     mu_run_test(test_get_factorial_3);
     mu_run_test(test_get_factorial_10);
     return 0;
 }
// ------- END TDD TESTS DEFINITION -------------

// Program entrypoint
int main(int argc, char **argv)
{

    if (argc == 2) {

        int number = atoi(argv[1]);
        int result = get_factorial(number);
        printf("%d\n", result);

    } else {
        printf("Usage: %s <number>\n", argv[0]);
    }

    return 0;
}
\end{lstlisting}

\begin{lstlisting}[language=c,caption={exercise\_5.c},captionpos=b]
/*
# Question:
# Write a funciton that works like a "Rock-Paper-Scissors" game, remember the
# rules:
# - Rock beats scissors
# - Scissors beats paper
# - Paper beats rock
#
# The result should be:
# - It's a tie!
# - Rock wins!
# - Scissors win!
# - Paper wins!
# - Invalid input!
#
# Suppose the following parameter is supplied to the program:
# rock scissors
# Then, the output should be:
# Rock wins!
*/

#include <stdio.h> //printf, scanf
#include <string.h>
#include "minunit.h"

int add(int x, int y) {
    return x + y;
}

char * play_game(char * option1, char * option2) {

    if (strcmp(option1, option2))
    {
        return("It's a tie!");
    }
    else if (strcmp(option1, "rock"))
    {
        if (strcmp(option2, "scissors"))
        {
            return("Rock wins!");
        }
        else
        {
            return("Paper wins!");
        }
    }
    else if (strcmp(option1, "scissors"))
    {
        if (strcmp(option2, "paper"))
        {
            return("Scissors win!");
        }
        else
        {
            return("Rock wins!");
        }
    }
    else if (strcmp(option1, "paper"))
    {
        if (strcmp(option2, "rock"))
        {
            return("Paper wins!");
        }
        else
        {
            return("Scissors win!");
        }
    }
    else
    {
        return("Invalid input!");
    }

}

// ------- START TDD TESTS DEFINITION -----------

char * test_play_game_rock_scissors(){
    mu_assert("error", strcmp(play_game("rock", "scissors"), "Rock wins!"));
    return 0;
}


char * test_play_game_scissors_scissors(){
    mu_assert("error", strcmp(play_game("scissors", "scissors"), "It's a tie!"));
    return 0;
}


char * test_play_game_paper_rock(){
    mu_assert("error", strcmp(play_game("paper", "rock"), "Paper wins!"));
    return 0;
}


char * test_play_game_scissors_paper(){
    mu_assert("error", strcmp(play_game("scissors", "paper"), "Scissors win!"));
    return 0;
}


char * test_play_game_invalid_paper(){
    mu_assert("error", strcmp(play_game("invalid", "paper"), "Invalid input!"));
    return 0;
}

char * all_tests() {
     mu_run_test(test_play_game_rock_scissors);
     mu_run_test(test_play_game_scissors_scissors);
     mu_run_test(test_play_game_paper_rock);
     mu_run_test(test_play_game_scissors_paper);
     mu_run_test(test_play_game_invalid_paper);

     return 0;
 }
// ------- END TDD TESTS DEFINITION -------------

// Program entrypoint
int main(int argc, char **argv)
{

    if (argc == 3) {

        char * option1 = argv[1];
        char * option2 = argv[2];
        char * result = play_game(option1, option2);
        printf("%s\n", result);

    } else {
        printf("Usage: %s <option1> <option2>\n", argv[0]);
    }

    return 0;
}
\end{lstlisting}

\subsection{Ejercicios HTML}

\begin{lstlisting}[language=html,caption={exercise\_1.html},captionpos=b]
<!--
Question:
Write a program that prints "Hello world!"

Hints:
Put your text directly inside the <body> tag
-->
<!DOCTYPE HTML PUBLIC "-//W3C//DTD HTML 4.01//EN" "http://www.w3.org/TR/html4/strict.dtd">
<html>
<head>
  <title>Hello World</title>
</head>
<body>
  Hello world!
</body>
</html>
\end{lstlisting}

\begin{lstlisting}[language=html,caption={exercise\_2.html},captionpos=b]
<!--
Question:

Create a simple page with a heading section and a paragraph
-->
<!DOCTYPE html>
<head>
    <title>exercise2</title>
</head>
<html>
<body>

<h1>My First Heading</h1>

<p>My first paragraph.</p>

</body>
</html>
\end{lstlisting}

\begin{lstlisting}[language=html,caption={exercise\_3.html},captionpos=b]
<!--
Question:
Create a html 4x4 html table and fill with data
-->
<!DOCTYPE html>
<html>
<head>
    <title>March Bills</title>
</head>

<body>

<h2>March Bills</h2>

<table summary="March bills">
    <tr>
        <th></th>
        <th>Price</th>
        <th>Due Date</th>
    </tr>

    <tr>
        <td>Phone</td>
        <td>$50</td>
        <td>March 1st</td>
    </tr>

    <tr>
        <td>Car insurance</td>
        <td>$100</td>
        <td>March 5th</td>
    </tr>

    <tr>
        <td>Internet</td>
        <td>$70</td>
        <td>March 10th</td>
    </tr>
</table>

</body>
</html>
\end{lstlisting}
