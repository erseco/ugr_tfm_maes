\addcontentsline{toc}{chapter}{Bibliografía}

% \subsubsection*{Libros consultados durante la realización del proyecto:}

% \printbibliography
\printbibliography[heading=bibempty]

\bigskip
\subsubsection*{Páginas de consulta sobre licencias, legislación y desarrollo y uso del software analizado}
\bibitem{CC} {\tt Creative Commons Share Alike 4.0}. \url{https://creativecommons.org/licenses/by-sa/4.0/}
\bibitem{rae} {\tt Diccionario RAE}. \url{http://dle.rae.es/}
\bibitem{boe} {\tt Boletín Oficial del Estado}. \url{https://www.boe.es/}
\bibitem{todofp} {\tt Todo FP}. \url{https://www.todofp.es/}
\bibitem{adide} {\tt ADIDE}. \url{https://www.adideandalucia.es/}
\bibitem{wikibooks} Wikibooks ({\tt LaTeX}). \url{https://en.wikibooks.org/wiki/LaTeX}
\bibitem{coderunner} {\tt Code Runner}. \url{https://github.com/trampgeek/moodle-qtype_coderunner/}
\bibitem{holamundo} {\tt Ejemplos del Hola Mundo}. \url{https://es.wikipedia.org/wiki/Anexo:Ejemplos_de_implementaci%C3%B3n_del_%C2%ABHola_mundo%C2%BB}
\bigskip
\subsubsection*{Otro material}
\begin{itemize}
	\item Diversas consultas puntuales al sitio {\tt Stack OverFlow}.
	\item Material docente de las asignaturas \textbf{Procesos y contextos educativos}, \textbf{Innovación docente e Investigación Educativa en Ciencia y Tecnología} y \textbf{Complementos de Formación} impartidas en el \master de la \textbf{Universidad de Granada}.
\end{itemize}