\chapter{Glosario de términos}

\textbf{Usabilidad}: facilidad con que las personas pueden utilizar una herramienta particular o cualquier otro objeto fabricado por humanos con el fin de alcanzar un objetivo concreto. La usabilidad es un término que no forma parte del diccionario de la Real Academia Española (RAE), aunque es bastante habitual en el ámbito de la informática y la tecnología.
\bigskip

\textbf{Accesibilidad}: grado en el que todas las personas pueden utilizar un objeto, visitar un lugar o acceder a un servicio, independientemente de sus capacidades técnicas, cognitivas o físicas. Es indispensable e imprescindible, ya que se trata de una condición necesaria para la participación de todas las personas independientemente de las posibles limitaciones funcionales que puedan tener.
\bigskip

\textbf{Seguridad de la información}: conjunto de medidas preventivas y reactivas de las organizaciones y de los sistemas tecnológicos que permiten resguardar y proteger la información buscando mantener la confidencialidad, la disponibilidad e integridad de datos y de la misma.
\bigskip

\textbf{Disponibilidad}: medida que nos indica cuánto tiempo está disponible ese equipo o sistema operativo respecto de la duración total durante la que se hubiese deseado que funcionase.
\bigskip

\textbf{MOOC}: Acrónimo en inglés de Massive Open Online Course, son cursos en línea dirigidos a un amplio número de participantes a través de Internet según el principio de educación abierta y masiva.
\bigskip

 \textbf{Auditoría de seguridad}: estudio que comprende el análisis y gestión de sistemas llevado a cabo por profesionales para identificar, enumerar y posteriormente describir las diversas vulnerabilidades que pudieran presentarse en una revisión exhaustiva de las estaciones de trabajo, redes de comunicaciones o servidores.
\bigskip

\textbf{Kahoot!}: Plataforma gratuita que permite la creación de cuestionarios de evaluación. Es una herramienta por la que el profesor crea una especie de concurso en el aula para aprender o reforzar el aprendizaje y donde los alumnos concursan desde su teléfono móvil.


\textbf{Blog}: es una contracción de web log. Los blogs son una forma de revista (journal) en línea usada por millones de personas en el mundo para expresarse a sí mismas y comunicarse con familiares y amigos.
\bigskip


\textbf{Frontend}: es la interfaz de la aplicación, es la parte de la aplicación que el usuario utiliza para comunicarse con la misma.
\bigskip

\textbf{Backend}: es el motor de una aplicación, se encarga de realizar las funciones en segundo plano que se encargan de que la aplicación funcione.
\bigskip

\textbf{URL (Uniform Resource Locator)}: nombre y con un formato estándar que permite acceder a un recurso de forma inequívoca.
\bigskip

\textbf{HTML (HyperText Markup Language)}: lenguaje de marcado que se utiliza para la realización de páginas web.
\bigskip

\textbf{JavaScript}: lenguaje de programación orientado a objetos interpretado que se utiliza principalmente para cargar programas desde el lado del cliente en los navegadores web.
\bigskip

\textbf{Expresión regular}: Una expresión regular, a menudo llamada también regex, es una secuencia de caracteres que forma un patrón de búsqueda, principalmente utilizada para la búsqueda de patrones de cadenas de caracteres u operaciones de sustituciones
\bigskip

\textbf{LaTeX}: sistema de composición de documentos que permite crear textos en diferentes formatos (artículos, cartas, libros, informes...) obteniendo una alta calidad en los documentos generados.
\bigskip

\textbf{Módulo}: fragmento de un programa desarrollado para realizar una tarea específica.
\bigskip

\textbf{Hash}: también llamadas funciones de resumen son algoritmos que consiguen crear a partir de una entrada (ya sea un texto, una contraseña o un archivo, por ejemplo) una salida alfanumérica de longitud normalmente fija que representa un resumen de toda la información que se le ha dado (es decir, a partir de los datos de la entrada crea una cadena que solo puede volverse a crear con esos mismos datos).
\bigskip

\textbf{Software libre}: software cuya licencia permite que este sea usado, copiado, modificado y distribuido libremente según el tipo de licencia que adopte.
\bigskip
