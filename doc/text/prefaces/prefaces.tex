\begin{center}
{\LARGE\bfseries\titulo}\\
\end{center}
\begin{center}
\autor\
\end{center}

\section*{Resumen}

\bigskip
\noindent{\textbf{Palabras clave}: \textit{\keywords}\\

Una de las mejores formas de aprender programación es a través de ejemplos. A su vez, la labor de corrección de ejercicios es una tarea que consume mucho tiempo y que puede dar lugar a multitud de errores.

\bigskip
Mediante el uso de sistemas de integración continua desarrollaremos un sistema de autocorrección de ejercicios de programación que asista a los profesores a la hora de corregir los mismos permitiendo al mismo tiempo que los alumnos puedan ver el resultado de dicha autocorrección para que les sirva como sistema de apoyo a su aprendizaje.

\newpage
\begin{center}
{\LARGE\bfseries\tituloEng}\\
\end{center}
\begin{center}
\autor\
\end{center}

\section*{Extended abstract}

\bigskip
\noindent{\textbf{Keywords}: \textit{\keywordsen}.\\

One of the best ways to learn computer programming is through examples. Also, the task of correcting exercises is a time-consuming task that can lead to a multitude of errors.

\bigskip
Through the use of continuous integration systems we are going to develop a system of self-correction of programming exercises that assists teachers in correcting them while allowing students to see the result of this self-correction so that it serves as a support system for their learning.

\bigskip
There are many studies (\cite{benotti_effect_2018}) and blogs posts like \cite{noauthor_how_2019} that demostrates how  self-correction and automatic code review systems improve programming learning by allowing students to self-evaluate immediately without having to wait for the correction of the exercises. In this way the student can see the correction of the exercise when he is working on it.

\bigskip
With the classic methods of manual correction the student must wait for this correction, days or even weeks can pass until seeing where it has failed and in many cases having lost the context of what the student was trying to solve. We are going to apply some previous research and knowledge (\cite{rubio_uso_2018}) to achieve the goal of building our custom auto-correction system.


\newpage
\thispagestyle{empty}
\
\vspace{3cm}

\noindent\rule[-1ex]{\textwidth}{2pt}\\[4.5ex]

Yo, \textbf{\autor}, alumno de la titulación \textbf{\master} de la \textbf{\escuela\ de la \universidad}, autorizo la ubicación de la siguiente copia de mi Trabajo Fin de Máster (\textit{\titulo}) en la biblioteca del centro para que pueda ser consultada por las personas que lo deseen.

\bigskip
Además, este mismo trabajo está publicado bajo la licencia \textbf{Creative Commons Attribution-ShareAlike 4.0} \cite{CC}, dando permiso para copiarlo y redistribuirlo en cualquier medio o formato, también de adaptarlo de la forma que se quiera, pero todo esto siempre y cuando se reconozca la autoría y se distribuya con la misma licencia que el trabajo original. Todo el código fuente así como este documento en formato {\tt LaTeX} se puede encontrar en el siguiente repositorio de {\tt GitHub}: \url{https://github.com/erseco/ugr_tfm_maes}.

\vspace{4cm}

\noindent Fdo: \autor

\vspace{2cm}

\begin{flushright}
\ciudad, a \today
\end{flushright}

\newpage
\thispagestyle{empty}
\
\vspace{3cm}

\noindent\rule[-1ex]{\textwidth}{2pt}\\[4.5ex]

D.ª \textbf{\tutor}, profesora del \textbf{Departamento de Lenguajes y Sistemas Informáticos} de la \textbf{\universidad}.

\vspace{0.5cm}

\vspace{0.5cm}

\textbf{Informa:}

\vspace{0.5cm}

Que el presente trabajo, titulado \textit{\textbf{\titulo}}, ha sido realizado bajo su supervisión por \textbf{\autor}, y
autoriza la defensa de dicho trabajo ante el tribunal que corresponda.

\vspace{0.5cm}

Y para que conste, expide y firma el presente informe en \ciudad\ a \today.

\vspace{1cm}

\textbf{La tutora:}

\vspace{3cm}

%\begin{figure}[H]
%\includegraphics[width=0.3\textwidth]{../../firmaJJ}
%\end{figure}

\noindent \textbf{\tutor}

\chapter*{Agradecimientos}
\thispagestyle{empty}

\vspace{1cm}

A Georgia, que algún día será mejor ingeniera que su tío.

\bigskip
Al ZX Spectrum 128K de mis hermanos, porque sin él no habría llegado hasta aquí.
