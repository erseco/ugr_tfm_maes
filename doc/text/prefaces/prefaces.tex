\begin{center}
{\LARGE\bfseries\titulo}\\
\end{center}
\begin{center}
\autor\
\end{center}

\section*{Resumen}

\bigskip
\noindent{\textbf{Palabras clave}: \textit{\keywords}\\

Una de las mejores formas de aprender programación es a través de la realización de ejercicios, pero la labor de corrección de dichos ejercicios es una tarea que consume mucho tiempo. El coste de la corrección manual de ejercicios tiene una correlación directa con el número de actividades que podemos realizar con nuestros alumnos ya que a la hora de planificarlas hay que tener en cuenta el tiempo que nos va a llevar corregir dichas actividades.

\bigskip
Con herramientas y/o metodologías que nos faciliten dicha corrección podríamos proponer muchos más ejercicios e incluso generar material de extensión para alumnos con altas capacidades. Los alumnos que tengan necesidades especiales de apoyo educativo podrían hacer uso de los resultados de la corrección para entender dónde fallan. Asimismo los propios alumnos pueden ayudarse mutuamente incentivando el trabajo en equipo.

\bigskip
Vamos a proponer una metodología de trabajo que mediante el uso de sistemas de integración continua permitirá la corrección automática de ejercicios de programación reduciendo así el tiempo de corrección manual. Además los alumnos podrán ver el resultado de dicha autocorrección sirviéndoles como sistema de apoyo a su aprendizaje.

\bigskip
Para ilustrar la metodología se van a desarrollar una serie de ejercicios típicos de programación de algunos lenguajes que están incluidos en el currículo de diversas asignaturas de Educación Secundaria, Bachillerato y Formación Profesional relacionadas con la informática.


\newpage
\begin{center}
{\LARGE\bfseries\tituloEng}\\
\end{center}
\begin{center}
\autor\
\end{center}

\section*{Extended abstract}

\bigskip
\noindent{\textbf{Keywords}: \textit{\keywordsen}.\\

One of the best ways to learn computer programming is through exercises but the work of correcting such exercises is a time-consuming task. The cost of the manual correction of exercises has a direct correlation with the number of activities that we can carry out with our students, since when planning them we have to take into account the time it will take us to correct these activities.

\bigskip
With tools and/or methodologies that facilitate this correction we could propose many more exercises and even generate extension material for students with high capacities. Students with special needs for educational support could use the results of the correction to understand where they fail. Students can also help each others encouraging the work in teams.

\bigskip
We are going to propose a working methodology that through the use of continuous integration systems will allow the automatic correction of programming exercises, thus reducing manual correction time. In addition, the students will be able to see the result of this self-correction, serving them as a support system for their learning.

\bigskip
We will develop some example exercises to show how works our methodology, these exercises will be developed in the most common programming languages that are included in the curriculum of various subjects related with the computer programming.

\bigskip
There are many studies (\cite{benotti_effect_2018}) and blogs posts like \cite{noauthor_how_2019} that demostrates how  self-correction and automatic code review systems improve programming learning by allowing students to self-evaluate immediately without having to wait for the correction of the exercises. In this way the student can see the correction of the exercise when he is working on it.

\bigskip
With the classic methods of manual correction the student must wait for this correction, days or even weeks can pass until seeing where it has failed and in many cases having lost the context of what the student was trying to solve. We are going to apply some previous research and knowledge (\cite{rubio_uso_2018}) to achieve the goal of building our custom auto-correction system.


\newpage
\thispagestyle{empty}
\
\vspace{10cm}

\noindent\rule[-1ex]{\textwidth}{2pt}\\[4.5ex]

% \section*{Declaración de Originalidad del TFM}

Yo, \textbf{\autor}, alumno de la titulación \textbf{\master} de la \textbf{\escuela\ de la \universidad}, declaro que el presente Trabajo de Fin de Máster es original, no habiéndose utilizado fuentes sin ser citadas debidamente. De no cumplir con este compromiso, soy consciente de que, de acuerdo con la Normativa de Evaluación y de Calificación de los estudiantes de la Universidad de Granada de 20 de mayo de 2013, \textit{esto conllevará automáticamente la calificación numérica de cero [...] independientemente del resto de las calificaciones que el estudiante hubiera obtenido. Esta consecuencia debe entenderse sin perjuicio de las responsabilidades disciplinarias en las que pudieran incurrir los estudiantes que plagien.}

\bigskip
Asimismo, autorizo la ubicación de la siguiente copia de mi Trabajo de Fin de Máster (\textit{\titulo}) en la biblioteca del centro para que pueda ser consultada por las personas que lo deseen.

\bigskip
Además, este mismo trabajo está publicado bajo la licencia \textbf{Creative Commons Attribution-ShareAlike 4.0} \cite{CC}, dando permiso para copiarlo y redistribuirlo en cualquier medio o formato, también de adaptarlo de la forma que se quiera, pero todo esto siempre y cuando se reconozca la autoría y se distribuya con la misma licencia que el trabajo original. Todo el código fuente así como este documento en formato {\tt LaTeX} se puede encontrar en los siguientes repositorios de {\tt GitHub}: \url{https://github.com/erseco/ugr_tfm_maes} y \url{https://github.com/erseco/ugr_tfm_maes_sample_exercises}.

\bigskip
Y para que así conste firmo el presente documento.

\vspace{3cm}

\noindent Fdo: \autor

\vspace{3cm}

\begin{flushright}
\ciudad, a \today
\end{flushright}

\newpage
\thispagestyle{empty}
\
\vspace{2cm}

\noindent\rule[-1ex]{\textwidth}{2pt}\\[4.5ex]

D.ª \textbf{\tutor}, profesora del \textbf{Departamento de Lenguajes y Sistemas Informáticos} de la \textbf{\universidad}.

\vspace{0.5cm}

\vspace{0.5cm}

\textbf{Informa:}

\vspace{0.5cm}

Que el presente trabajo, titulado \textit{\textbf{\titulo}}, ha sido realizado bajo su supervisión por \textbf{\autor}, y
autoriza la defensa de dicho trabajo ante el tribunal que corresponda.

\vspace{0.5cm}

Y para que conste, expide y firma el presente informe en \ciudad\ a \today.

\vspace{1cm}

\textbf{La tutora:}

\vspace{3cm}

%\begin{figure}[H]
%\includegraphics[width=0.3\textwidth]{../../firmaZoraida}
%\end{figure}

\noindent \textbf{\tutor}

\chapter*{Agradecimientos}
\thispagestyle{empty}

\vspace{1cm}

A Georgia, que algún día será mejor ingeniera que su tío.

\bigskip
Al ZX Spectrum 128K +2A de mis hermanos, porque sin él no habría llegado hasta aquí.

\bigskip
A mi \textit{seño} Zoraida, por haber sido mi primera profesora en la carrera, por ponerme mi primera matrícula y por cerrar el círculo dirigiendo este TFM.
