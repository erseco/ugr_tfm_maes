\documentclass[spanish,a4paper,11pt]{book}
% \usepackage[spanish]{babel}
\usepackage[utf8]{inputenc}

% \usepackage{cite} % para contraer referencias

\usepackage{listings}
\usepackage{titlesec}
\usepackage{fancyhdr}
\usepackage[spanish]{babel}
\usepackage[utf8]{inputenc}

\usepackage{xcolor}
\usepackage{pdfpages}
\usepackage{url}
\usepackage{booktabs}
\usepackage[export]{adjustbox}
\usepackage{fancybox}

% \usepackage{biblatex-apa}

\usepackage[backend=biber,style=apa]{biblatex}
% \usepackage[style=numeric-comp,citestyle=apa,sorting=nyt,sortcites=true,autopunct=true,babel=hyphen,hyperref=true,abbreviate=false,backref=true,backend=biber]{biblatex}
% \usepackage{biblatex}


\usepackage[babel]{csquotes}


% \usepackage[backend=biber,style=apa]{biblatex}
\DeclareLanguageMapping{spanish-apa}
% \addbibresource{bibliography/project.bib}


% %----------------------------------------------------------------------------------------
%	BIBLIOGRAPHY AND INDEX
%----------------------------------------------------------------------------------------

% \usepackage[style=alphabetic,citestyle=numeric,sorting=nyt,sortcites=true,autopunct=true,babel=hyphen,hyperref=true,abbreviate=false,backref=true,backend=biber]{biblatex}

% \usepackage[style=numeric-comp,citestyle=ieee,sorting=nyt,sortcites=true,autopunct=true,babel=hyphen,hyperref=true,abbreviate=false,backref=true,backend=biber]{biblatex}

\addbibresource{bibliography/project.bib} % BibTeX bibliography file
\defbibheading{bibempty}{}




\usepackage{textcomp}

\usepackage{wrapfig}

\usepackage{blindtext}

\usepackage{float}

\usepackage{booktabs}

\usepackage{rotating}

\usepackage[hidelinks]{hyperref}


% Información reutilizable
\newcommand{\asunto}{Trabajo de Fin de Máster}
\newcommand{\titulo}{Autocorrección interactiva para la enseñanza y aprendizaje de la programación}
\newcommand{\tituloEng}{Interactive autocorrection for the teaching and learning of programming}
\newcommand{\master}{Máster en Profesorado de Enseñanza Secundaria Obligatoria y Bachillerato, Formación Profesional y Enseñanzas de Idiomas}
\newcommand{\autor}{Ernesto Serrano Collado}
\newcommand{\email}{info@ernesto.es}
\newcommand{\tutor}{Zoraida Callejas Carrión}
\newcommand{\escuela}{Escuela Internacional de Posgrado}
\newcommand{\universidad}{Universidad de Granada}
\newcommand{\ciudad}{Granada}
\newcommand{\vers}{Versión 0.1}
\providecommand{\keywords}{educación, enseñanza, aprendizaje, programación, integración continua, sistemas de control de versiones, software libre}

\providecommand{\keywordsen}{education, teaching, learning, programming, continuous integration, version control systems, free software}


% Información archivo
\hypersetup{
	pdfauthor = {\autor\ (\email)},
	pdftitle = {\titulo},
	pdfsubject = {\asunto},
	pdfkeywords = {\keywords},
	pdfcreator = {MacTeX con el paquete TeX Live},
	pdfproducer = {pdflatex}
}

% Estilo de cabeceras
\pagestyle{fancy}
\fancyhf{}
\fancyhead[LO]{\leftmark}
\fancyhead[RE]{\rightmark}
\fancyhead[RO,LE]{\textbf{\thepage}}
\setlength{\headheight}{1.5\headheight}

% Redefinición de comandos
\renewcommand{\lstlistingname}{Fragmento de código}
\renewcommand{\lstlistlistingname}{Índice de fragmentos de código}
\renewcommand{\chaptermark}[1]{\markboth{\textbf{#1}}{}}
\renewcommand{\sectionmark}[1]{\markright{\textbf{\thesection. #1}}}

% Definición de colores
\definecolor{gray97}{gray}{.97}
\definecolor{gray75}{gray}{.75}
\definecolor{gray45}{gray}{.45}
\definecolor{gray30}{gray}{.94}
\definecolor{lightgray}{rgb}{.9,.9,.9}
\definecolor{darkgray}{rgb}{.4,.4,.4}
\definecolor{purple}{rgb}{0.65, 0.12, 0.82}
\definecolor{background}{HTML}{EEEEEE}
\definecolor{delim}{RGB}{20,105,176}
\colorlet{punct}{red!60!black}
\colorlet{numb}{magenta!60!black}

	\definecolor{dkgreen}{rgb}{0,0.6,0}
	\definecolor{gray}{rgb}{0.5,0.5,0.5}
	\definecolor{mauve}{rgb}{0.58,0,0.82}

% Listados
\lstset{
	aboveskip=0.5cm,
	backgroundcolor=\color{gray97},
	basicstyle=\scriptsize\ttfamily,
	breaklines=true,
	%commentstyle=\color{gray45},
	frame=Ltb,
	framerule=0.5pt,
	framesep=0pt,
	framexbottommargin=3pt,
	framexleftmargin=0.1cm,
	framextopmargin=3pt,
	%keywordstyle=\bfseries,
	numberfirstline = false,
	numbers=left,
	numbersep=6pt,
	%numberstyle=\tiny,
	rulesep=.4pt,
	rulesepcolor=\color{black},
	showstringspaces = false,
	%stringstyle=\ttfamily,
	  numberstyle=\tiny\color{gray},
	  keywordstyle=\color{blue},
	  commentstyle=\color{dkgreen},
	  stringstyle=\color{mauve},
	literate={á}{{\'a}}1
	         {é}{{\'e}}1
	         {í}{{\'i}}1
	         {ó}{{\'o}}1
	         {ú}{{\'u}}1
	         {ñ}{{\~n}}1
}


% Minimizar fragmentado de listados
\lstnewenvironment{listing}[1][]
	{\lstset{#1}\pagebreak[0]}{\pagebreak[0]}

% Listado definido para JavaScript
% http://tex.stackexchange.com/questions/89574/language-option-supported-in-listings/89576#89576
\lstdefinelanguage{javascript}{
	backgroundcolor=\color{background},
	basicstyle=\footnotesize,
	breaklines=true,
	captionpos=b,
	comment=[l]{//},
	commentstyle=\color{purple}\ttfamily,
	frame=lines,
	identifierstyle=\color{black},
	keywordstyle=\color{blue}\bfseries,
	morecomment=[s]{/*}{*/},
	morestring=[b]',
	morestring=[b]",
	ndkeywordstyle=\color{darkgray}\bfseries,
	numbers=left,
	numbersep=8pt,
	numberstyle=\scriptsize,
	sensitive=false,
	showstringspaces=false,
	stepnumber=1,
	stringstyle=\color{red}\ttfamily,
	keywords={
		break,
		case,
		catch,
		catch,
		do,
		else,
		false,
		function,
		if,
		in,
		new,
		null,
		return,
		switch,
		true,
		typeof,
		var,
		while},
	ndkeywords={
		boolean,
		class,
		export,
		implements,
		import,
		this,
		throw}
}

% Listado definido para JSON
% http://tex.stackexchange.com/questions/83085/how-to-improve-listings-display-of-json-files/83100#83100
\lstdefinelanguage{json}{
	backgroundcolor=\color{background},
	basicstyle=\footnotesize,
	breaklines=true,
	captionpos=b,
	frame=lines,
	numbers=left,
	numbersep=8pt,
	numberstyle=\scriptsize,
	showstringspaces=false,
	stepnumber=1,
	literate=
		*{:}{{{\color{punct}{:}}}}{1}
		{,}{{{\color{punct}{,}}}}{1}
	    {\{}{{{\color{delim}{\{}}}}{1}
	    {\}}{{{\color{delim}{\}}}}}{1}
	    {[}{{{\color{delim}{[}}}}{1}
	    {]}{{{\color{delim}{]}}}}{1}
	    {ñ}{{\~{n}}}{1}
}

\lstdefinelanguage{yaml}{
  keywords={true,false,null,y,n},
  keywordstyle=\color{darkgray}\bfseries,
  ndkeywords={},
  ndkeywordstyle=\color{black}\bfseries,
  identifierstyle=\color{black},
  sensitive=false,
  %moredelim=[l]{}{:},
  comment=[l]{#},
  morecomment=[s]{/*}{*/},
  commentstyle=\color{purple}\ttfamily,
  stringstyle=\color{blue}\ttfamily,
  %morestring=[l]{-}{},
  morestring=[b]',
  morestring=[b]"
}

% Para que las páginas en blanco no tengan cabecera
\makeatletter
\def\clearpage{%
  \ifvmode
    \ifnum \@dbltopnum =\m@ne
      \ifdim \pagetotal <\topskip
        \hbox{}
      \fi
    \fi
  \fi
  \newpage
  \thispagestyle{empty}
  \write\m@ne{}
  \vbox{}
  \penalty -\@Mi
}
\makeatother

\begin{document}
\input{front/front}
\frontmatter
\begin{center}
{\LARGE\bfseries\titulo}\\
\end{center}
\begin{center}
\autor\
\end{center}

\section*{Resumen}

\bigskip
\noindent{\textbf{Palabras clave}: \textit{\keywords}\\

Una de las mejores formas de aprender programación es a través de ejemplos. A su vez, la labor de corrección de ejercicios es una tarea que consume mucho tiempo y que puede dar lugar a multitud de errores.

\bigskip
Mediante el uso de sistemas de integración continua desarrollaremos un sistema de autocorrección de ejercicios de programación que asista a los profesores a la hora de corregir los mismos permitiendo al mismo tiempo que los alumnos puedan ver el resultado de dicha autocorrección para que les sirva como sistema de apoyo a su aprendizaje.

\newpage
\begin{center}
{\LARGE\bfseries\tituloEng}\\
\end{center}
\begin{center}
\autor\
\end{center}

\section*{Extended abstract}

\bigskip
\noindent{\textbf{Keywords}: \textit{\keywordsen}.\\

One of the best ways to learn computer programming is through examples. Also, the task of correcting exercises is a time-consuming task that can lead to a multitude of errors.

\bigskip
Through the use of continuous integration systems we are going to develop a system of self-correction of programming exercises that assists teachers in correcting them while allowing students to see the result of this self-correction so that it serves as a support system for their learning.

\bigskip
There are many studies (\cite{benotti_effect_2018}) and blogs posts like \cite{noauthor_how_2019} that demostrates how  self-correction and automatic code review systems improve programming learning by allowing students to self-evaluate immediately without having to wait for the correction of the exercises. In this way the student can see the correction of the exercise when he is working on it.

\bigskip
With the classic methods of manual correction the student must wait for this correction, days or even weeks can pass until seeing where it has failed and in many cases having lost the context of what the student was trying to solve. We are going to apply some previous research and knowledge (\cite{rubio_uso_2018}) to achieve the goal of building our custom auto-correction system.


\newpage
\thispagestyle{empty}
\
\vspace{3cm}

\noindent\rule[-1ex]{\textwidth}{2pt}\\[4.5ex]

Yo, \textbf{\autor}, alumno de la titulación \textbf{\master} de la \textbf{\escuela\ de la \universidad}, autorizo la ubicación de la siguiente copia de mi Trabajo Fin de Máster (\textit{\titulo}) en la biblioteca del centro para que pueda ser consultada por las personas que lo deseen.

\bigskip
Además, este mismo trabajo está publicado bajo la licencia \textbf{Creative Commons Attribution-ShareAlike 4.0} \cite{CC}, dando permiso para copiarlo y redistribuirlo en cualquier medio o formato, también de adaptarlo de la forma que se quiera, pero todo esto siempre y cuando se reconozca la autoría y se distribuya con la misma licencia que el trabajo original. Todo el código fuente así como este documento en formato {\tt LaTeX} se puede encontrar en el siguiente repositorio de {\tt GitHub}: \url{https://github.com/erseco/ugr_tfm_maes}.

\vspace{4cm}

\noindent Fdo: \autor

\vspace{2cm}

\begin{flushright}
\ciudad, a \today
\end{flushright}

\newpage
\thispagestyle{empty}
\
\vspace{3cm}

\noindent\rule[-1ex]{\textwidth}{2pt}\\[4.5ex]

D.ª \textbf{\tutor}, profesora del \textbf{Departamento de Lenguajes y Sistemas Informáticos} de la \textbf{\universidad}.

\vspace{0.5cm}

\vspace{0.5cm}

\textbf{Informa:}

\vspace{0.5cm}

Que el presente trabajo, titulado \textit{\textbf{\titulo}}, ha sido realizado bajo su supervisión por \textbf{\autor}, y
autoriza la defensa de dicho trabajo ante el tribunal que corresponda.

\vspace{0.5cm}

Y para que conste, expide y firma el presente informe en \ciudad\ a \today.

\vspace{1cm}

\textbf{La tutora:}

\vspace{3cm}

%\begin{figure}[H]
%\includegraphics[width=0.3\textwidth]{../../firmaJJ}
%\end{figure}

\noindent \textbf{\tutor}

\chapter*{Agradecimientos}
\thispagestyle{empty}

\vspace{1cm}

A Georgia, que algún día será mejor ingeniera que su tío.

\bigskip
Al ZX Spectrum 128K de mis hermanos, porque sin él no habría llegado hasta aquí.


\begingroup
\let\cleardoublepage\clearpage
  \tableofcontents
 % \listoffigures
%  \listoftables
 % \lstlistoflistings
\endgroup

\newpage
\thispagestyle{empty}
\
\mainmatter
\chapter{Introducción}

\section{Motivación}

Desde que era pequeño me ha llamado la atención el concepto enseñanza. El aprender...

compartir conocimientos
enseñar lo que se sabe




\blindtext

\section{Definición del problema}

\blindtext

Este proyecto intentará resolver los siguientes problemas:

\begin{itemize}
  \item Evitar la tediosa tarea de la corrección de ejercicios.
  \item Motivar la enseñanza y apredizaje de la programación a través de ejemplos.
  \item Ver como un sistema de control de versiones es una herramienta básica que debería enseñarse a la vez que se enseña programación.
  \item Mejorar la calidad de vida tanto de profesores como de estudiantes
\end{itemize}

\blindtext

\section{Estructura del proyecto}


\bigskip
Antes de pasar a detalles más técnicos, me gustaría detallar el contenido de este proyecto:

\begin{itemize}
  \item En el \textit{capítulo 1} (\textbf{Introducción}) se encuentra un texto descriptivo del proyecto, así como una breve introducción a las tecnologías que se utilizarán.
  \item El \textit{capítulo 2} (\textbf{Objetivos}) detalla de forma algo más concreta los objetivos determinados que se quieren cumplir con este proyecto.
  \item En el \textit{capítulo 3} (\textbf{Metodología}) está la planificación y desarrollo de cada uno de los apartados del proyecto.
  \item En el \textit{capítulo 4} (\textbf{Resultados})se detallan todos los resultados obtenidos.
  \item En el \textit{capítulo 5} (\textbf{Conclusiones}) se pueden encontrar las conclusiones finales  así como las recomendaciones para futuros trabajos.

\end{itemize}

\bigskip
Para finalizar se incluye un anexo con el código fuente desarrollado y liberado bajo una licencia libre.


% %
% % Ejemplos de codigo LaTeX para uso futuro
% %

% \begin{figure}[h!]
% \centering
% \includegraphics{../screenshots/sample1}
% \caption{Sample Image 1}
% \label{fig:sample1}
% \end{figure}

% Esto es un texto con una nota\footnote{Ejemplo de nota al pie} al pie.

% Y esto es una ``Frase de alguien''\cite{stevekrug}.

% \begin{itemize}
%   \item \textbf{1.} Texto de ejemplo
%   \item \textbf{2.} Texto de ejemplo
%   \item \textbf{3.} Texto de ejemplo
%   \item \textbf{4.} Texto de ejemplo

% \end{itemize}

% \begin{enumerate}
% 	\item Ejemplo 1.
% 	\item Ejemplo 2.
% \end{enumerate}

% \begin{lstlisting}[language=html]
% <!DOCTYPE html>
% <html lang="es-ES">
%   <head>
%     <meta charset="utf-8">
%     <title>Ejemplo de 2 párrafos</title>
%   </head>
%   <body>
%     <p>Esto es un párrafo.</p>
%     <p>Esto es otro párrafo.</p>
%   </body>
% </html>
% \end{lstlisting}

% Puedes verlo en \cite{Patricio2011}. Te recomiendo leer \cite{Patricio2011, Zacarias2009, Alfonso2010b, Alfonso2010a}.

\chapter{Objetivos}

El objetivo máximo de este proyecto es la elaboración de un sistema software que permita la autocorrección de ejercicios de programación.

\bigskip
Dicho objetivo se descompone los siguientes objetivos principales:

\begin{itemize}
  \item \textbf{OBJ-1.} Desarrollar una herramienta para la autocorrección de ejercicios de programación.
  \item \textbf{OBJ-2.} Analizar el estado del arte actual en cuanto a herramientas de autocorrección.
  \item \textbf{OBJ-3.} Analizar las posibilidades de los sistemas de autocorrección actualmente analizados.
  \item \textbf{OBJ-4.} Determinar la ventaja que nos brindan los sistemas de autocorrección para mejorar la ensañanza y aprendizaje de la programación.
\end{itemize}

Además como objetivos secundarios tendremos:

\begin{itemize}
  \item \textbf{OBJ-5.} Desarrollar una serie de ejercicios de ejemplo para comprobar el correcto funcionamiento de la herramienta.
  \item \textbf{OBJ-6.} Definir los usos posibles de la herramienta tanto para la enseñanza como para el aprendizaje de la programación.
\end{itemize}


\section{Alcance de los objetivos}
El fin inmediato de este informe es desarrollar una herramienta que permita a los profesores definir una serie de ejercicios que los propios alumnos puedan corregir de forma autonoma y automática de una forma sencilla.
Además todo el código así como la documentación resultante se liberará con una licencia libre para que cualquiera pueda hacer de la herramienta desarrollada así como de las conclusiones obtenidas.

\section{Interdependencia de los objetivos}

Todos los objetivos son interdependientes entre sí, pero el primer objetivo (\textbf{OBJ-1}) es el principal motivador de este proyecto, por lo que aún sin representar el desarrollo de ningún trabajo en concreto es el que va a escudar y avalar el desarrollo de los demás. En aspectos más relacionados con la realización del proyecto, el tercer objetivo (\textbf{OBJ-3}) es el que nos brindará el sistema sobre la que trabajar, ya que sienta la base sobre la que aplicar el cuarto objetivo (\textbf{OBJ-4}). El resto de objetivos secundarios, al no tener un carácter urgente serán resueltos en base a la disponibilidad del tiempo necesario para su realización.

\section{Conocimientos y herramientas utilizadas}

\bigskip
Destacar en los aspectos formativos previos más utilizados para el desarrollo del proyecto los conocimientos adquiridos en las asignaturas ``Cloud Computing'' para el análisis y configuración de los diferentes servicios de red así como todo lo referente a virtualización de sistemas, ``Programación Orientada a Objetos'' para el desarrollo de los ejercicios de ejemplo, ``Planificación y Gestión de. Proyectos Informáticos.'' para definir los requisitos y el planteamiento inicial del proyecto, ``Seguridad en Sistemas Operativos'' para lo referente a las buenas prácticas de programación y ``Servidores Web de Altas Prestaciones'' para la realización de pruebas desde el punto de vista de disponibilidad y carga de trabajo.

Para la realización de cada una de las partes se han usado multitud de herramientas específicas tales como pueden ser \texttt{Docker}, \texttt{Travis} \texttt{GitHub}, \texttt{LaTeX} y \texttt{git} entre otras.

\chapter{Antecedentes}

En este capítulo vamos analizar el estado de arte actual y las tecnologías candidatas a utilizarse en este proyecto en base a los objetivos que presentamos en el capítulo anterior.

\section {Herramientas de corrección de codigo}

\subsection {Coderunner}

\blindtext

\subsection {Pythontutor}

\blindtext

\section {...}

\blindtext




\chapter{Propuesta Pedagógica}

En este capítulo pasamos a detallar la ventaja pedagógica que se pretende obtener en en base a los objetivos que presentamos en el segundo capítulo y a los antecedentes vistos en el tercero.


\section{Análisis de las soluciones existentes}

Tras analizar las tres soluciones libres descritas en los antecedentes (Coderunner, Python Tutor y Jupyter Notebooks) hemos encontrado que las tres tienen el mismo problema. En todos los casos nuestros alumnos tienen que aprender a usar una herramienta concreta que posiblemente no vayan a volver a utilizar en su futuro profesional. Esto sería comparable a lo que didácticamente decimos que el alumno ``aprende a aprobar exámenes'' mas que aprender los contenidos de la asignatura. Por esto, nos gustaría proponer una metodología donde los alumnos usen herramientas que van a tener que usar en futuro profesional (o académico) como pueden ser un sistema de control de versiones así como sistemas de integración continua, o \textit{continuous integration (CI)}.

\section {Temario relacionado con la programación informática}

Tal y como indica el Decreto 110/2016  de 14 de junio, por el que se establece la ordenación y el currículo del Bachillerato en la Comunidad Autónoma de Andalucía las tecnologías de la información y de la comunicación para el aprendizaje y el conocimiento se utilizarán de manera habitual como herramientas integradas para el desarrollo del currículo. Asimismo en los currículos de las distintas ramas educativos donde podemos ejercer como profesores incluyen asignaturas relacionadas con la informática y de forma más específica con la programación informática.

\bigskip
Tecnologías de la Información y Comunicación es un término amplio que enfatiza la integración de la informática y las telecomunicaciones, y de sus componentes hardware y software, con el objetivo de garantizar a
los usuarios el acceso, almacenamiento, transmisión y manipulación de información. Su adopción y generalización han provocado profundos cambios en todos los ámbitos de nuestra vida, incluyendo la educación, la sanidad, la democracia, la cultura y la economía, posibilitando la transformación de la Sociedad Industrial en la Sociedad del Conocimiento.

\subsection {Educación Secundaria Obligatoria (ESO)}

En la Orden de 14 de julio de 2016 se especifica el currículo de la materia de ``Tecnologías de la Información y Comunicación'' que es una materia de opción del bloque de asignaturas específicas para el alumnado de cuarto curso de la Educación Secundaria Obligatoria. Es una asignatura de carácter optativo que intenta dar una visión global de la informática, con lo que su temario intenta abarcar desde historia de la informática hasta la seguridad o la programación web pasando por la ofimática y estudio de los elementos de hardware, nuestra metodología propuesta podría servir de ayuda para la enseñanza de programación de páginas web con HTML que es uno de los contenidos que se dan en dicha asignatura.

\subsection {Bachillerato}

En la Orden de 14 de julio de 2016, por la que se desarrolla el currículo correspondiente al Bachillerato en la Comunidad Autónoma de Andalucía indica las diferentes modalidades de bachillerato, incluyendo las modalidades de Ciencias y la de Humanidades y Ciencias Sociales las asignaturas de carácter obligatorio ``Tecnologías de la Información y de la Comunicación I'' y ``Tecnologías de la Información y de la Comunicación II'' y habiendo también una asignatura de libre configuración autonómica llamada ``Programación y Computación''. Estas asignaturas contienen contenidos de programación informática y de programación de páginas web por lo tanto son susceptibles de utilizar nuestra metodología de realización de ejercicios.

\subsection {Ciclo Formativo Grado Medio}

La Orden EDU/2187/2009, de. 3 de Julio establece el currículo del ciclo formativo de Grado Medio correspondiente al título de ``Técnico en Sistemas Microinformáticos y Redes (SMR)'' que es el único ciclo de grado medio relacionado con la informática donde la única asignatura ligeramente relacionada con la programación es ``Aplicaciones web'', pero dicha asignatura se centra más en el despliegue de aplicaciones que en la programación en sí, por lo que aunque nuestra metodología se podría adaptar para enseñar a configurar como desplegar aplicaciones web no entra dentro del alcance de este trabajo por lo que se deja como posible trabajo futuro.

\subsection {Ciclo Formativo Grado Superior}

Existen tres ciclos formativos de grado superior del plan LOE relacionados con la informática, la Orden EDU/392/2010, de 20 de enero, por la que se establece el currículo del ciclo formativo de Grado Superior correspondiente al título de ``Técnico Superior en Administración de Sistemas Informáticos en Red'' describe las asignaturas ``Lenguajes de Marca y Sistemas de Gestión de Información'' y ``Gestión de Bases de Datos'' en las que nuestra metodología se puede aplicar para corregir los ejercicios ya que se enseñan los lenguajes HTML, SQL, Javascript, XML así como algunos subconjuntos de XML como son DTD y XSD. Los currículos del título de ``Técnico Superior en Desarrollo de Aplicaciones Web'' definido en la Orden EDU/2887/2010, de 2 de noviembre y el de ``Técnico Superior en Desarrollo de Aplicaciones Multiplataforma'' definido en la Orden EDU/2000/2010, de 13 de julio definen varias asignaturas, algunas comunes, como pueden ser ``Programación'', ``Bases de Datos'' y ``Lenguajes de marcas y sistemas de gestión de información'' y otras específicas de cada título como pueden ser ``Programación multimedia y dispositivos móviles'' o ``Desarrollo web en entorno cliente'' que están centradas en el aprendizaje de la programación informática y en los que si encajaría nuestra metodología.


\subsection {Formación Profesional Básica}

La Orden ECD/1030/2014, de 11 de junio define el ``Título Profesional Básico en Informática y Comunicaciones'' y la Orden ECD/1633/2014, de 11 de septiembre define el  ``Título Profesional Básico en Informática de Oficina'', siendo ambas titulaciones las únicas relacionadas con la informática de todas las que componen la formación profesional básica, pero al igual que con el ciclo formativo de Grado Medio ambos currículos carecen de temario específico de programación por lo que queda fuera del ámbito de este proyecto.


\section{Encuesta de percepción}

Se realizó una breve encuesta y se difundió entre algunos profesores de diferentes ámbitos con el fin de ver si nuestra hipótesis de que la corrección de ejercicios era la parte mas tediosa que realizaban como docentes.

\bigskip
Aprovechamos la encuesta para obtener también algunos datos interesantes sobre la forma de trabajar de los docentes, de cara a trabajos futuros. La encuesta se podía consultar en la siguiente dirección web: \url{https://forms.gle/xxSz6mQDjswo5B137}.

\bigskip
La encuesta estuvo activa desde el 19 hasta el 25 de Mayo de 2019 y se recopilaron un total de 47 respuestas. Puede parecer una población pequeña pero se compartió ``de boca a oreja'' indicando que por sólo lo compartieran con otros profesores a ser posible relacionados con la informática para no desvirtuar las respuestas.


\begin{enumerate}

\item \textbf{¿Dónde sueles impartir tus clases?}

Con esta pregunta queríamos tener una visión de la población encuestada, ya que al difundirla entre profesores y animarles a compartirla entre sus compañeros profesores era muy posible que sus red de contactos no se limitara a profesores de Educación Secundaria, así que dejamos elegir entre las siguientes opciones:

\begin{itemize}
    \item Educación Primaria
    \item Educación Secundaria
    \item Universidad (Grado/Máster)
    \item Practicas MAES
\end{itemize}

\bigskip
La gran mayoría de los encuestados han resultado ser profesores de Educación Secundaria, por lo que los resultados han sido satisfactorios, los siguientes han sido estudiantes en prácticas en el \master los cuales podríamos englobar dentro del conjunto de los de Secundaria, llegando así casi al 90\% de la población encuestada. Le siguen la respuestas de algunos profesores de Universidad y por último algunos maestros de Educación Primaria.

\begin{figure}[H]
\centering
\includegraphics[width=1.0\textwidth]{../images/quiz_1}
\caption{¿Dónde sueles impartir tus clases?}
\label{fig:quiz_1}
\end{figure}


\item \textbf{Valora las siguientes tareas que sueles realizar}

Esta es quizá la pregunta más importante de toda la encuesta, porque queríamos saber la opinión de diversos profesores sobre las tareas de corrección de ejercicios y/o exámenes. Además nos ha servido para hacernos una idea de lo que opinan de las tareas más comunes de un profesor, había que valorarlas según se consideraban ``divertidas'', ``normales'' o ``tediosas'' entendiéndose tediosas como aburridas. Las opciones eran las siguientes:

\begin{itemize}
    \item Preparación de material
    \item Preparación de clases
    \item Impartir clases
    \item Corrección de ejercicios
    \item Evaluación final
    \item Tutorías
\end{itemize}

Como podemos ver, la hipótesis de que corregir ejercicios resulta tedioso ha sido validada. De hecho ha sido la única de las opciones que no ha recibido ningún voto como actividad ``divertida''. Del resto de preguntas podemos deducir que a la mayoría de profesores les divierte impartir clases, esto es algo normal. No tendría sentido una respuesta diferente dado el carácter vocacional de la profesión, y si nos centráramos en los aspectos económicos es mas flagrante en el mundo de la informática porque los sueldos de la empresas privadas están muy por encima de lo que se puede llegar a cobrar como docente. El resto de preguntas se han considerado normales, habiendo opiniones en las tres vertientes. Quizá remarcar que la evaluación es la que menos votos de tarea ``divertida'' han recibido. Aunque esto puede variar mucho de un centro a otro.

\begin{figure}[H]
\centering
\includegraphics[width=1.0\textwidth]{../images/quiz_2}
\caption{Valora las siguientes tareas que sueles realizar}
\label{fig:quiz_2}
\end{figure}

\item \textbf{¿Elaboras tus propios ejercicios?}

En esta pregunta queríamos saber si los profesores elaboran sus propios ejercicios. En mi experiencia como alumno así como en mi experiencia como docente en prácticas he visto exactamente lo que se ve reflejado en la encuesta, hay profesores que elaboran todo su material, los hay que lo hacen de forma regular pero no siempre, también los hay que los realizan de forma ocasional. Aquí me gustaría remarcar que la opción ``Nunca'' ha recibido muy pocos votos cuando en mi experiencia también hay una mucho docentes que utilizan el material de otros profesores en lugar del suyo propio.

\begin{figure}[H]
\centering
\includegraphics[width=1.0\textwidth]{../images/quiz_3}
\caption{¿Elaboras tus propios ejercicios?}
\label{fig:quiz_3}
\end{figure}

\item \textbf{¿Crees que hay que compartir los ejercicios con el resto de la comunidad?}

Esta pregunta era para ver la repercusión que tiene a nivel docente los conceptos del software libre y las licencias abiertas como la GPL y la ``Creative Commons'' \cite{comons_creative_2013} así como la WikiPedia y otros proyectos que abogan por la difusión libre y gratuita tanto de software como de todo tipo de contenidos culturales.

\bigskip
Por nuestra experiencia pensábamos que no iba a haber tan buena acogida del ``Si'' como ha habido, pero nos ha sorprendido gratamente que la gran mayoría de los profesores encuestados apoyen el compartir el material de apoyo. También es algo que se puede considerar normal ya que hoy en día es difícil hacer uso de recursos de terceros encontrados online para apoyar nuestras clases. De hecho en mi periodo como docente en prácticas me sirvió para ver que al contrario que en mi época de estudiante, donde usábamos el libro de texto, los docentes usan recursos como Kahoot!\footnote{Plataforma gratuita que permite la creación de cuestionarios de evaluación en forma de concurso.} o YouTube para hacer las clases mucho mas dinámicas.

\begin{figure}[H]
\centering
\includegraphics[width=1.0\textwidth]{../images/quiz_4}
\caption{¿Crees que hay que compartir los ejercicios con el resto de la comunidad?}
\label{fig:quiz_4}
\end{figure}

\item \textbf{¿Sueles entregar correcciones de los ejercicios?}

Esta pregunta está íntimamente relacionada con el propósito de este trabajo, queríamos saber si los profesores solían entregar los ejercicios corregidos, hay que tener en cuenta que para que dicha corrección tenga utilidad las correcciones se han de entregar de forma temprana, ya que es cuando el alumno aun está adquiriendo los conocimientos, de dieron las opciones ``Si'', ``No'' y ``Otra'', siendo esta última opción para que los profesores indicaran alguna opción distinta, a continuación se pega en bruto dichas opciones alternativas:

\begin{itemize}
    \item Algunas veces
    \item A veces y otras se corrigen en clase.
    \item Sí o se corrigen en clase.
\end{itemize}

Nos pareció interesante la opción de la corrección en clase, pues no la habíamos tenido en cuenta, pensamos que es una forma muy didáctica de que el alumno vea donde ha fallado y hacer una corrección temprana, pero teniendo en cuenta que por norma general siempre hay una sensación de falta de tiempo para impartir el temario, así que quizá invertir tiempo en esta corrección puede hacer que lo perdamos para impartir nuevos contenidos.

\begin{figure}[H]
\centering
\includegraphics[width=1.0\textwidth]{../images/quiz_5}
\caption{¿Sueles entregar correcciones de los ejercicios?}
\label{fig:quiz_5}
\end{figure}

\item \textbf{¿Qué opinas de los sistemas de corrección automática?}

Aquí preguntamos de forma directa sobre los sistemas de corrección automática de forma genérica, sin especificar ninguno, ya que como hemos visto en los antecedentes existen diversos métodos, aquí les dimos a elegir entre estas opciones:

\begin{itemize}
    \item Ahorran tiempo a alumnos y profesores
    \item Mejoran la enseñanza al poder ver de manera más inmediata los resultados
    \item Desvirtúan la finalidad de los ejercicios propuestos
    \item Deshumaniza la relación profesor-alumno
    \item Otra (especificar)
\end{itemize}

La opción ``Mejoran la enseñanza al poder ver de manera más inmediata los resultados'' fue marcada por casi el 50\% de los encuestados, seguida por la de ``Ahorran tiempo a alumnos y profesores'' que consiguió algo mas del 20\%, ambas en conjunto eran las opciones que eran más acordes al desarrollo de nuestra metodología por lo que podemos ver que hay un claro interés en estos sistemas, además pusimos la opción ``Otra'' donde los profesores indicaron opiniones adicionales que pasamos a enumerar:

\begin{itemize}
    \item Ahorran tiempo a todos y mejoran la enseñanza. Se pueden hacer más ejercicios y practicar mucho es esencial en enseñanzas técnicas.
    \item Sirven para ver si los alumnos han adquirido los conocimientos.
    \item Es otro tipo de herramienta de evaluación
    \item No lo he probado
    \item Son útiles en su justa medida
\end{itemize}

En estas respuestas vemos que en su mayoría también coinciden que los sistemas de corrección automática son muy útiles para mejorar la enseñanza, así que creemos que esta propuesta metodológica puede ser de gran ayuda a profesores.

\begin{figure}[H]
\centering
\includegraphics[width=1.0\textwidth]{../images/quiz_6}
\caption{¿Qué opinas de los sistemas de corrección automática?}
\label{fig:quiz_6}
\end{figure}

\item \textbf{Sugerencias para mejorar las clases}

Incorporamos a la encuesta un apartado para que los profesores aportaran sus sugerencias para mejorar las clases, a continuación vamos a poner los datos recopilados en bruto en dicho apartado:

\begin{itemize}
    \item Hacerlas dinámicas y participativas, huir de la explicación meramente.
    \item Lo más eficiente para el aprendizaje del alumnado es que ellos mismos corrijan, así si se les indican los errores suelen verlos y evitarlos después.
    \item Que parte de la evaluación la hagan los propios alumnos (son más conscientes de los errores cometidos).
    \item Fomentar el trabajo basado en proyectos colaborativos.
    \item Correcciones en clase de la totalidad de los ejercicios, haciendo partícipes a los alumnos.
    \item Coevaluación, aula invertida. Cooperativo.
    \item Participación continuada de alumno.
    \item La tipología de ejercicio más eficiente a la hora de su corrección es el tipo test. Existe una optimización de tiempo de corrección bastante notable en comparación con ejercicios con preguntas de desarrollo, donde se tarda más por no tener tan automatizada la respuesta correcta (hay que leer la respuesta), la puntuación se somete a matices por estar incompleta, en ocasiones es difícil comprender la letra del alumno, etc.
\end{itemize}


\end{enumerate}

\section{Definición de la metodología}

Tras validar que nuestra propuesta efectivamente tiene un alto valor pedagógico y que encaja dentro de los contenidos de varias asignaturas de los currículos de educación secundaria pasamos a definir los objetivos, contenidos y competencias de nuestra metodología.

\subsection{Objetivos}

El objetivo de nuestra metodología es enseñar programación de una forma diferente, con ejercicios, autocorrección y con tecnologías que se usan en el día a día de cualquier informático que se dedique profesionalmente a ello.

\bigskip
Como indica el Decreto 110/2016  de 14 de junio el alumno debe ``conocer el funcionamiento de las nuevas tecnologías de la información y la comunicación, comprendiendo sus fundamentos y utilizándolas para el tratamiento de la información (buscar, almacenar, organizar, manipular, recuperar, presentar, publicar y compartir), así como para la elaboración de programas que resuelvan problemas tecnológicos''


\subsection{Contenidos}

Los contenidos de nuestra metodología comprenderán dos partes. Una parte común a todas las asignaturas, que podrá obviarse o abreviarse si los alumnos ya tienen dichas competencias adquiridas ya sea por haberlo aprendido en otra asignatura o haberlo aprendido de forma autodidacta. Otra parte específica donde haciendo uso de lo aprendido en la parte común seguirán el temario indicado para la asignatura en cuestión.

\bigskip
La parte común comprenderá los siguientes dos apartados que se definirán de modo mas extenso en la propuesta metodológica:

\begin{enumerate}
    \item Aprendiendo a usar Git
    \item Introducción a la programación
\end{enumerate}

La parte específica se desarrollará en base a los contenidos de la asignatura a impartir, pero en la propuesta metodológica se definirán algunos ejercicios de ejemplo de diferentes asignaturas como pueden ser ``Lenguajes de marcas y sistemas de gestión de información'', ``Programación'' o ``Bases de Datos''.


\subsection{Competencias}

La programación informática, y por extensión nuestra metodología, permite adquirir las siguientes 5 del total de 7 competencias clave de la LOMCE descritas en la Orden ECD/65/2015, de 21 de enero:

\begin{itemize}
    \item \textbf{CCL}: Competencia en comunicación lingüística
    \item \textbf{CMCT}: Competencia matemática y competencias básicas en ciencia y tecnología
    \item \textbf{CD}: Competencia Digital
    \item \textbf{CPAA}: Competencia de aprender a aprender
    \item \textbf{SIE}: Sentido de la iniciativa y espíritu emprendedor
\end{itemize}

Entrando en detalle, de nuestra metodología, podemos destacar y definir las siguientes tres:

\begin{itemize}

    \item Utilizar las tecnologías de la información y la comunicación de modo habitual en el proceso de aprendizaje, buscando, analizando y seleccionando información relevante en Internet o en otras fuentes,  elaborando documentos propios, haciendo exposiciones y argumentaciones de los mismos y compartiendo éstos en entornos apropiados para facilitar la interacción.

    \item Conocer el funcionamiento de las nuevas tecnologías de la información y la comunicación, comprendiendo sus fundamentos y utilizándolas para el tratamiento de la información (buscar, almacenar, organizar, manipular, recuperar, presentar, publicar y compartir), así como para la elaboración de programas que resuelvan problemas tecnológicos.

    \item Asumir de forma crítica y activa el avance y la aparición de nuevas tecnologías, incorporándolas al quehacer cotidiano.

\end{itemize}





\chapter{Propuesta Metodológica}

En este capítulo vamos a definir y a detallar los procesos que vamos a seguir para llevar a cabo la propuesta pedagógica definida en el capítulo anterior.

\section{Herramientas necesarias}

\subsection{Sistemas de control de versiones}

Lo primero que necesitaremos es enseñar a nuestros alumnos a utilizar un sistema de control de versiones, aunque existen varios sistemas (SVN, Mercurial, CVS), hoy en día estándar de facto es Git que además es software libre.

Para una administración más sencilla de nuestro código utilizaremos un cliente web como es GitHub, aunque existen otros como pueden ser GitLab o BitBucket.

\subsubsection {Git}

Git es un sistema de control de versiones distribuido que nos ayuda a llevar un seguimiento de los cambios en el código mientras desarrollamos. Se diseñó para su uso con código fuente pero se puede utilizar para llevar el seguimiento de los cambios en cualquier conjunto de archivos. Sus características incluyen la velocidad, la integridad de los datos y la compatibilidad con flujos de trabajo distribuidos y no lineales.

Git fue creado por Linus Torvalds en 2005 para el desarrollo del núcleo de Linux como alternativa al software BitKeeper que era un sistema de control de versiones propietario tras cambios en la licencia de este último.

Como con la mayoría de los sistemas de control de versiones distribuidos, y a diferencia de la mayoría de los sistemas cliente-servidor cada directorio de Git es un repositorio completo con un historial completo y capacidades de seguimiento de versiones completas, independiente del acceso a la red o a un servidor central.

Git es software libre y de código abierto distribuido bajo los términos de la Licencia Pública General GNU versión 2.

\subsubsection {GitHub}

GitHub es un servicio de alojamiento basado en web para el control de versiones a través de Git, fue adquirido por MicroSoft en 2018. Se utiliza sobre todo para código fuente aunque se puede utilizar para almacenar todo tipo de contenidos, incluso libros. Ofrece toda la funcionalidad de control de versiones distribuido y gestión de código fuente de Git, además de añadir sus propias características.

Proporciona control de acceso y varias funciones de colaboración como seguimiento de errores, \textit{pull requests}, gestión de tareas y \textit{wikis} para cada proyecto. Además cuenta con un sub-proyecto llamado GitHub Education que es ampliamente utilizado como herramientas para la formación de programadores (\cite{hernandez_integracion_2018}).


\section {Sistemas de integración continua (CI)}

\subsubsection {Travis-CI}

Travis CI es un servicio alojado de integración continua que se utiliza para construir y testear proyectos de software alojados en GitHub.

Los proyectos de código abierto pueden ser probados de forma gratuita a través del sitio \url{https://travis-ci.org}. En cambio los proyectos privados deben pagar para poder ser ejecutados a través del sitio \url{https://travis-ci.com}.

Gran parte de su código fuente es software libre y está disponible en GitHub.

\subsection{Lenguajes utilizados}

Al igual que no existe una única solución a un problema, existen multitud de lenguajes de programación, nosotros nos vamos a centrar en solamente cuatro de ellos ya que son los que cumplen con los contenidos de la mayoría de las asignaturas sobre informática, dichos lenguajes son Python, Ruby, C y HTML, aun así esta metodología es aplicable a otros lenguajes por lo que se podría utilizar para enseñar Java, Go, Haskell o Javascript:

\subsubsection{Python}

Python es un lenguaje de programación desarrollado por Guido van Rossum en 1991 cuya sintaxis favorece escribir código muy legible. Hoy en día es uno de los lenguajes de programación más utilizados en cursos de introducción a la programación ya que su sintaxis suele ser más sencilla, asemejándose al ``pseudocódigo''.

Según diversos índices (\cite{TIOBE2019}) y encuestas (\cite{stack_overflow_stack_2019}) Python es uno de los lenguajes con mayor proyección de futuro y que del que más se ha incrementado su uso, sobre todo motivado por su sencillez a la hora de aprenderlo y a su potencia como lenguaje de análisis de datos habiendo prácticamente desbancando al lenguaje estadístico R.

\bigskip
Vamos a centrar nuestros ejercicios en la versión 3 de Python ya que la versión 2 actualmente solo recibe actualizaciones de seguridad y en enero de 2020 dejará de estar oficialmente soportada (\cite{python.org_pep_2018}).

\subsubsection{Ruby}

Ruby es un lenguaje de programación orientado a objetos creado por el programador japonés Yukihiro Matsumoto en 1995. Su orientación a objetos es denominada ``fuerte'' ya que todos los tipos de dato son a su vez un objeto.

\bigskip
Ruby empezó a ganar popularidad tras la publicación del framework de aplicaciones web Ruby On Rails (RoR) por parte de David Heinemeier Hansson en 2005 ya que el mismo simplificaba muchísimo el desarrollo siguiendo el patrón Modelo Vista Controlador (MVC).

\bigskip
Debido a su sencilla sintaxis y su fuerte orientación a objetos es ampliamente utilizado para enseñar las particularidades de dicho paradigma.

\subsubsection{C}

El lenguaje de programación C es un lenguaje de propósito general desarrollado por Dennis Ritchie en los Laboratorios Bell entre 1969 y 1972, es el lenguaje de programación más popular para crear software para sistemas embebidos y micro-controladores, aunque también se puede utilizar para crear aplicaciones.

\bigskip
Es un lenguaje tipado estáticamente (en tiempo de compilación) y considerad de de medio nivel ya que dispone de las estructuras típicas de los lenguajes de alto nivel pero permitiendo un control a muy bajo nivel. Los compiladores suelen ofrecer extensiones al lenguaje que posibilitan mezclar código en ensamblador con código C, acceder a memoria o controlar diferentes dispositivos.

\subsubsection{HTML}

El lenguaje de marcas de hipertexto o HTML por sus siglas en inglés, es un lenguaje de marcado para la elaboración de páginas web. La primera versión del lenguaje fue creada por Tim Berners-Lee en 1991 mientras trabajaba en el Centro Europeo de Investigaciones Nucleares (CERN) en Suiza. Actualmente sus especificaciones están a cargo del World Wide Web Consortium (W3C).

\bigskip
HTML no es un lenguaje de programación, es un lenguaje de marcado que sirve para definir documentos estandarizados.

\section {Herramientas de análisis de código}

También conocidos como ``linters''  son herramientas que analizan el código fuente para marcar errores de programación, \textit{bugs}, errores estilísticos y código sospechosos. Algunos son capaces incluso de calcular la complejidad ciclomática\footnote{Valor escalar que mide la complejidad lógica de un programa.} de un algoritmo. El término proviene de una antigua utilidad de Unix para análisis de código fuente en lenguaje C llamada ``lint''.

\bigskip
Existen \textit{linters} para prácticamente todos los lenguajes de programación, en nuestros ejemplos hemos utilizado los siguientes:

\subsection {Pycodestyle}

Pycodestyle es una herramienta para comprobar código fuente en el lenguaje Python contra las convenciones de estilo PEP8.

\bigskip
PEP8 es una guía de estilo de codificación para Python definida inicialmente por Guido van Rossum, creador del lenguaje Python y que está considerada como la estándar del lenguaje.

\subsection{Rubocop}

Es un analizar de código para el lenguaje Ruby que tiene opciones muy interesantes como el cálculo de la complejidad ciclomática y la auto-reparación de código fuente incorrecto.

\subsection{Cpplint}

Herramienta de código abierto desarrollada por Google, diseñada para garantizar que el código C y C++ se ajusta a las guías de estilo de codificación de la compañía.

\subsection {HTML Tidy}

Es una aplicación de consola que sirve para corregir código HTML no válido, detectar posibles errores de accesibilidad web y mejorar el diseño y el estilo de sangría del marcado resultante. Fue desarrollado en 2002 por el miembro del World Wide Web Consortium (W3C) Dave Raggett.

\section {Herramientas de verificación de código}

Las herramientas de verificación de código se utilizan para ejecutar los test TDD o BDD en nuestro código fuente. En nuestros ejercicios vamos a utilizar las siguientes  :

\subsection {Pytest}

Pytest es un framework de test unitarios para el lenguaje de programación Python que facilita la creación de pruebas sencillas y escalables. Las pruebas son expresivas y legibles y no se requiere código adicional.

\subsection {RSpec}

RSpec es una herramienta para realizar test BDD que sirve para probar código escrito en el lenguaje de programación Ruby.

\subsection{MinUnit}

MinUnit es un mini-framework para correr test unitarios en lenguaje C. Su código fuente se puede encontrar en \url{http://www.jera.com/techinfo/jtns/jtn002.html}. Es particularmente pequeño ya que consiste en un fichero de cabecera ``.h'' de tan solo 4 líneas de código.

\section{Metodología de aprendizaje}

Algunas de las herramientas necesarias para nuestro sistema de corrección requieren algo de formación previa, por lo que vamos a definir una metodología de aprendizaje para las mismas

\subsection{Aprendiendo a usar Git}

Como ya explicamos anteriormente Git es uno de los sistemas de control de versiones más utilizados en la actualidad. Es difícil encontrar hoy en día una empresa que se dedique a la programación de forma profesional que no la utilice. A pesar de ello no forma parte del currículo de los ciclos formativos de informática. De hecho, salvo contadas excepciones, tampoco se aprende su uso en los Grados en Ingeniería Informática. De ahí la ventaja de nuestro sistema, ya que vamos a aprovechar para enseñarles a utilizar una herramienta que van a usar de forma exhaustiva durante su futura carrera profesional.

\bigskip
Para aprender a utilizar Git nos vamos a basar en diversos manuales que enseñan a utilizarlo de una forma sencilla (\cite{popov_control_2012}) sin necesidad de tener conocimientos de programación. Como ya indicamos en la propuesta pedagógica vamos a definir unos contenidos comunes que son los siguientes:

\subsubsection{Introducción a Git}

En esta parte les explicaremos que es un sistema de control de versiones, que diferencias hay entre un sistema centralizado y uno distribuido. También les hablaremos de la motivación de por qué usarlos poniéndoles ejemplos de lo que seguramente ellos han utilizado hasta ahora que seguramente será duplicar los archivos en distintas carpetas. También les hablaremos un poco de la historia de Git.

\subsubsection{Creando una cuenta en GitHub}

Aquí les indicaremos como crear su primera cuenta en GitHub, para ello deberán acceder a la página web \url{https://github.com} y hacer clic en SignUp (figura \ref{fig:git1}).

\begin{figure}[H]
\centering
\includegraphics[width=1.0\textwidth]{../images/git1}
\caption{Pagina de registro de GitHub}
\label{fig:git1}
\end{figure}

\subsubsection{Creando nuestro primer repositorio}

Crear un repositorio en GitHub es muy intuitivo, solo tenemos que hacer clic en el botón New y rellenar los datos que nos solicita (figura \ref{fig:git2}). De igual forma podemos crear un repositorio de forma local en nuestro ordenador ejecutando el comando \texttt{git init}

\begin{figure}[H]
\centering
\includegraphics[width=1.0\textwidth]{../images/git2}
\caption{Ventana de creación de repositorio en GitHub}
\label{fig:git2}
\end{figure}

Para que los alumnos se familiaricen con el uso de GitHub editaremos el fichero \textcc{README.md} con el editor WYSIWYG\footnote{acrónimo de What You See Is What You Get (en español, "lo que ves es lo que obtienes").} integrado en la plataforma.

\subsubsection{Comandos básicos de Git}

Aquí podemos ver algunos de los comandos más habituales de Git, se puede encontrar la referencia completa en su documentación oficial en \url{https://git-scm.com/doc}.

\begin{itemize}
  \item \textbf{git clone}: Clona un repositorio
  \item \textbf{git status}: Nos dice el estado de un repositorio
  \item \textbf{git commit}: Nos permite guardar los cambios en una rama
  \item \textbf{git checkout}: Nos permite cambiar de rama
  \item \textbf{git branch}: Nos permite crear y listar ramas
  \item \textbf{git push}: Permite enviar el código a un repositorio remoto
  \item \textbf{git pull}: Permite obtener el código desde un repositorio remoto
\end{itemize}

Para que los alumnos se familiaricen con el uso de estos comandos vamos a pedirle que hagan cambios en los archivos de la carpeta \textcc{homework_00_markdown} con su editor y hagan un \textit{commit} y un \textit{push} con el código fuente.

\subsubsection{Creando nuestro primer Fork}

Una bifurcación, o fork en inglés, es el término que se utiliza para indicar una ramificación de un trabajo. Básicamente significa que vamos a copiar un proyecto y crear uno nuevo haciéndole modificaciones. La capacidad de crear bifurcaciones de código de forma sencilla es una de las características que han ayudado a la plataforma GitHub llegar a ser el sitio de referencia para albergar proyectos de software libre.

\bigskip
Para crear un fork en GitHub de cualquier proyecto simplemente tenemos que hacer click en el botón situado a la derecha de cada proyecto (figura \ref{fig:git_fork}). Una cosa a tener en cuenta a la hora de hacer un Fork de un proyecto es la licencia bajo la que esté dicho código, el cual suele venir indicado en el fichero LICENSE, no todas las licencias permiten la libre distribución de proyectos derivados.

\begin{figure}[H]
\centering
\includegraphics[width=1.0\textwidth]{../images/git_fork}
\caption{Detalle del botón de Fork en GitHub}
\label{fig:git_fork}
\end{figure}

\subsubsection{Creando un Pull-Request}

Las contribuciones a un repositorio de código fuente que utiliza un sistema de control de versiones distribuido se realizan comúnmente por medio de un ``pull request''. El colaborador solicita que el encargado del proyecto haga un ``pull'' con los cambios en el código fuente, de ahí el nombre. El mantenedor puede revisar el conjunto de cambios, discutir modificaciones potenciales o mezclar el código.

Dependiendo del flujo de trabajo establecido el código puede ser probado antes de ser incluido en la versión oficial. Algunos proyectos ejecutan un conjunto de pruebas automatizadas en cada solicitud de extracción, utilizando una herramienta de integración continua como Travis CI, y el revisor verifica que cualquier código nuevo tenga la cobertura de pruebas adecuada.

Para hacer un ``Pull request'' haremos clic en el botón ``New pull request'' y seleccionando que ramas queremos fusionar (figura \ref{fig:git_pr}).

\begin{figure}[H]
\centering
\includegraphics[width=1.0\textwidth]{../images/git_pr}
\caption{Detalle de creación de un Pull Request en GitHub}
\label{fig:git_pr}
\end{figure}

\subsubsection{Sistemas de integración continua (CI)}

Un sistema de integración continua suele consistir en una plataforma que ejecuta una serie de pasos con cada ``Push'' que realizamos a nuestro sistema de control de versiones. A esta serie de pasos se le suele denominar ``pipeline'' y suele contener pasos habituales como pueden ser la compilación, la ejecución de validaciones sintácticas y estilísticas de código (\textit{linter}) y la ejecución de los diferentes test para comprobar que efectivamente el código funciona.

\bigskip
Como ya hemos indicado, para nuestra metodología vamos a utilizar ``Travis CI'' aunque la forma de funcionar es muy similar en casi todas las plataformas. Para crear una cuenta en ``Travis CI'' iremos a la url \url{https://travis-ci.org} y haremos click en el botón ``Sign-Up'' (figura \ref{fig:travis_signup})).

\begin{figure}[H]
\centering
\includegraphics[width=1.0\textwidth]{../images/travis_signup}
\caption{Detalle de creación de una cuenta en Travis CI}
\label{fig:travis_signup}
\end{figure}

Una vez tengamos nuestra cuenta tenemos que activar en cuales de nuestra lista de repositorios queremos activar el servicio. Para ellos solo debemos hacer click en el interruptor hasta que quede de color verde (figura \ref{fig:travis_enable}).

\begin{figure}[H]
\centering
\includegraphics[width=1.0\textwidth]{../images/travis_enable}
\caption{Detalle de activación de un repositorio en Travis CI}
\label{fig:travis_enable}
\end{figure}

Una vez activado, con cada \texttt{git push} se ejecutará el ``pipeline'' definido en el fichero \texttt{.travis.yml}, en nuestro repositorio de ejemplo se puede obtener uno configurado para ejecutar los diferentes ejemplos realizados para mostrar el funcionamiento de la metodología.

\subsection{Introducción a la programación}

Una vez los alumnos han aprendido los comandos básicos de Git y se han dado de alta en GitHub es hora de enseñarles algunos conceptos básicos de programación.

\subsubsection{Conceptos básicos de programación}

Todos los lenguajes de programación comparten algunos elementos básicos que funcionan y se usan de forma diferente en cada lenguaje, pero que cumplen el mismo objetivo. Esos elementos son:

\begin{itemize}
  \item Tipos de datos: Enteros, Decimales, Caracteres, Cadenas de texto, etc...
  \item Variables: donde almacenar los datos.
  \item Control de flujo: los ``if''
  \item Bucles: los conocidos ``loops'', ``for'' y ``while''
  \item Funciones
  \item Entra y Salida: pintando en pantalla.
\end{itemize}

Hay que remarcar que estamos explicando los conceptos básicos de programación, por eso esto no incluye ni estructuras de datos, ni orientación a objetos ni recursividad, realmente eso lo aprenderán en ejercicios adicionales.

\subsubsection{Nuestro primer ``Hola Mundo''}

Un ``Hola mundo'' es un programa cuya única finalidad es escribir la frase ``Hola mundo!''. Este programa se usa como introducción en la mayoría de lenguajes de programación siendo el primer ejercicio típico, y se considera fundamental desde el punto de vista didáctico. Una implementación de dicho programa se puede encontrar para prácticamente todos los lenguajes de programación existentes.

\bigskip
En nuestra metodología animamos a los profesores a enseñar un primer ejemplo del lenguaje a impartir usando un ``Hola Mundo'' para que los alumnos se familiaricen con el lenguaje. En nuestros ejercicios de ejemplo vamos a incorporar el hola mundo como primer ejercicio de programación.

\subsubsection{Guías de estilo}

En esta sección les enseñaremos la guía de estilo que vamos a seguir. Aunque es algo que se suele obviar en cursos de programación, a la hora de escribir código de calidad es importante seguir una guía de estilo, además de crear buenas prácticas de programación. Les explicaremos lo importante que es comentar correctamente y se utilizará a poder ser una guía de estilo estandarizada para el lenguaje que vayamos a impartir.

\bigskip
Si el lenguaje que vamos a impartir no tuviera guía de estilo estandarizada, o si el profesor lo estima didáctico se podría consensuar con los alumnos el tipo de estilo que se va a seguir. Este caso es aplicable al uso de comillas simples '' o dobles "" pues no hay un consenso sobre cuales se deben usar, igual pasa con la indentación con espacios o con tabuladores.

\subsubsection{Introducción al TDD}

Aquí les explicaremos en consiste el TDD, que como ya hemos visto es escribir la prueba, escribir el código y una vez funcione refactorizar. En nuestro caso concreto les vamos a dar nosotros realizadas las pruebas TDD, pero les podremos animar a que realicen pruebas adicionales.

\subsubsection{Ejercicios de ejemplo}

En el repositorio \url{https://github.com/erseco/ugr_tfm_maes_sample_exercises} hemos definido una serie de ejercicios así como sus correcciones, dichos ejercicios se han incorporado en la sección de Anexos.

\subsection{Usando la integración continua para aprender}

Una vez hemos visto como usar Git y los conceptos básicos de programación vamos a ver algunos de los errores con los que se pueden encontrar los alumnos, como pueden usar el sistema de integración continua para detectarlos ellos mismos y como corregirlos (figura \ref{fig:linter_error_python}).

\begin{figure}[H]
\centering
\includegraphics[width=1.0\textwidth]{../images/linter_error_python}
\caption{Detalle de un error de linter python en un pipeline}
\label{fig:linter_error_python}
\end{figure}



\chapter{Conclusiones}

Tras el análisis de las numerosas herramientas de corrección y verificación existentes, así como la retroalimentación recibida parte de profesores y alumnos, sin olvidar a las personas que respondieron a la encuesta realizada he concluido que, a pesar de tener algunos inconvenientes, los sistema de autocorrección nos proveen de una serie de ventajas muy a tener en cuenta.

Aun así, dichos sistemas tienen algunos inconvenientes. \textit{Coderunner}, por ejemplo, depende de que tengamos una instalación de Moodle, y \textit{Pythontutor} está limitado al aprendizaje del lenguaje Python.

Además, dichos sistemas existentes, necesitan de un aprendizaje previo.

Nuestro sistema tiene la ventaja de utilizar tecnologías estándares con amplia difusión en la industria informática con lo que no van a aprender a usar una herramienta que no usarán jamás cuando finalicen la asignatura, todo lo contrario, les vamos a enseñar a usar tecnologías con las que se van a tener que enfrentar más tarde o más temprano.

En mis años de Universidad jamás me hablaron de los \textit{linters}, de las guías de estilo de codificación y de las pruebas unitarias (TDD) se vio de pasada. Tampoco aparece el uso de estas herramientas en ninguna de las asignaturas relacionadas con la informática de Educación Secundaria, Bachillerato y Formación Profesional. Hay una asignatura llamada ``Entornos de desarrollo'' que enseña a usar algunos IDEs\footnote{Entorno de Desarrollo Integrado, Integrated Development Environment en inglés.} no contempla ninguna de las tecnologías mencionadas.

Esta metodología sirve para enseñar cualquier lenguaje de programación o paradigma de programación existente, es decir, es completamente agnóstica del lenguaje.

Al guardarse un registro de todo lo que se envía al repositorio de código junto al resultado de su compilación, el profesor puede tener métricas muy detalladas de los errores cometidos por cada alumno, ver su evolución e incluso el tiempo dedicado.

Observando el \textit{pipeline} los profesores pueden ver de forma global donde están fallando los alumnos y realizar modificaciones sobre los ejercicios para incidir en los temas en los que los alumnos tengan mayor dificultad.

Además, al ser software libre y estar disponible de forma pública cualquiera puede hacer uso de los ejercicios e incluso proponer mejoras, a su vez los alumnos pueden enseñar su código en futuras entrevistas de trabajo. A día de hoy el mejor currículum vitae de un programador informático es su perfil en GitHub.

Como inconvenientes podemos destacar que gran parte de los docentes tampoco están habituados a usar los sistemas utilizados por lo que deberían formarse en su utilización. Además la curva de trabajo para el profesor es más pronunciada pues tiene que preparar todos los ejercicios  gran variedad de ejercicios y test antes del comienzo de las clases.

Otro inconveniente es que el profesor debe ir modificando regularmente los ejercicios ya que al estar disponibles de forma pública una simple búsqueda les permitiría a los alumnos obtener la resolución de los ejercicios de cursos anteriores.

Aun con estos inconvenientes vemos que las ventajas son mucho mayores, y sumándole la retroalimentación recibida parte de profesores y alumnos además de las personas que respondieron a la encuesta podemos concluir que nuestro sistema aporta un valor añadido evidente a los sistemas existentes de corrección y espero implantar la metodología en mis propias clases.

\chapter{Anexos}

\section{Código fuente}

\subsection{Pipeline para Travis-CI}

\begin{lstlisting}[language=yaml]
---
dist: xenial

before_install:
  - pyenv global system 3.6.7
  - sudo apt-get update
  - sudo apt-get install -y tidy
install:
  - pip3 install pytest pycodestyle

script:
  - echo "Testing that we have all the requirements"
  - python3 --version
  - ruby --version
  - gcc --version
  - tidy --version

  - pycodestyle homework_01/*.py
  - pytest --verbose homework_01/*.py

  - tidy -quiet -errors homework_04/*.html
\end{lstlisting}

\subsection{Ejercicios Python}
\begin{lstlisting}[language=python]
def add(x, y):
    return x + y


# Prueba la función con los valores (4, 5)
# Resultado: 9 (True)
def test_add():
    assert add(4, 5) == 9
\end{lstlisting}

\begin{lstlisting}[language=python]
import pytest


def capital_case(x):
    if not isinstance(x, str):
        raise TypeError('Please provide a string argument')
    return x.capitalize()


def test_capital_case():
    assert capital_case('semaphore') == 'Semaphore'


def test_raises_exception_on_non_string_arguments():
    with pytest.raises(TypeError):
        capital_case(9)
\end{lstlisting}

\subsection{Ejercicios Ruby}

\subsection{Ejercicios C++}

\subsection{Ejercicios HTML}

\begin{lstlisting}[language=html]
<!DOCTYPE html>
<html lang="es-ES">
  <head>
    <meta charset="utf-8">
    <title>Ejemplo de 2 párrafos</title>
  </head>
  <body>
    <p>Esto es un párrafo.</p>
    <p>Esto es otro párrafo.</p>
  </body>
</html>
\end{lstlisting}



% \backmatter
% \begingroup
% 	\setlength\parindent{0pt}
% 	\chapter{Glosario de términos}

\textbf{Usabilidad}: facilidad con que las personas pueden utilizar una herramienta particular o cualquier otro objeto fabricado por humanos con el fin de alcanzar un objetivo concreto. La usabilidad es un término que no forma parte del diccionario de la Real Academia Española (RAE), aunque es bastante habitual en el ámbito de la informática y la tecnología.
\bigskip

\textbf{Accesibilidad}: grado en el que todas las personas pueden utilizar un objeto, visitar un lugar o acceder a un servicio, independientemente de sus capacidades técnicas, cognitivas o físicas. Es indispensable e imprescindible, ya que se trata de una condición necesaria para la participación de todas las personas independientemente de las posibles limitaciones funcionales que puedan tener.
\bigskip

\textbf{Seguridad de la información}: conjunto de medidas preventivas y reactivas de las organizaciones y de los sistemas tecnológicos que permiten resguardar y proteger la información buscando mantener la confidencialidad, la disponibilidad e integridad de datos y de la misma.
\bigskip

\textbf{Disponibilidad}: medida que nos indica cuánto tiempo está disponible ese equipo o sistema operativo respecto de la duración total durante la que se hubiese deseado que funcionase.
\bigskip


\textbf{MOOC}: Acrónimo en inglés de Massive Open Online Course, son cursos en línea dirigidos a un amplio número de participantes a través de Internet según el principio de educación abierta y masiva.
\bigskip

 \textbf{Auditoría de seguridad}: estudio que comprende el análisis y gestión de sistemas llevado a cabo por profesionales para identificar, enumerar y posteriormente describir las diversas vulnerabilidades que pudieran presentarse en una revisión exhaustiva de las estaciones de trabajo, redes de comunicaciones o servidores.
\bigskip


\textbf{SCORM}: (del inglés Sharable Content Object Reference Model) es un conjunto de estándares y especificaciones que permite crear objetos pedagógicos estructurados. Los sistemas de gestión de contenidos en web originales usaban formatos propietarios para los contenidos que distribuían.
\bigskip


\textbf{Blog}: es una contracción de web log. Los blogs son una forma de revista (journal) en línea usada por millones de personas en el mundo para expresarse a sí mismas y comunicarse con familiares y amigos.
\bigskip

\textbf{Fundación Mozilla}: organización sin ánimo de lucro que produce software libre.
\bigskip

\textbf{Frontend}: es la interfaz de la aplicación, es la parte de la aplicación que el usuario utiliza para comunicarse con la misma.
\bigskip

\textbf{Backend}: es el motor de una aplicación, se encarga de realizar las funciones en segundo plano que se encargan de que la aplicación funcione.
\bigskip

\textbf{URL (Uniform Resource Locator)}: nombre y con un formato estándar que permite acceder a un recurso de forma inequívoca.
\bigskip

\textbf{HTML (HyperText Markup Language)}: lenguaje de marcado que se utiliza para la realización de páginas web.
\bigskip

\textbf{JavaScript}: lenguaje de programación orientado a objetos interpretado que se utiliza principalmente para cargar programas desde el lado del cliente en los navegadores web.
\bigskip

\textbf{Python}: lenguaje de programación interpretado cuya filosofía hace hincapié en una sintaxis que favorezca un código legible.
\bigskip

\textbf{JSON (JavaScript Object Notation)}: formato de texto plano usado para el intercambio de información, independientemente del lenguaje de programación.
\bigskip

\textbf{Expresión regular}: Una expresión regular, a menudo llamada también regex, es una secuencia de caracteres que forma un patrón de búsqueda, principalmente utilizada para la búsqueda de patrones de cadenas de caracteres u operaciones de sustituciones
\bigskip

\textbf{LaTeX}: sistema de composición de documentos que permite crear textos en diferentes formatos (artículos, cartas, libros, informes...) obteniendo una alta calidad en los documentos generados.
\bigskip

\textbf{SSH (Secure SHell)}: protocolo que permite conectarse a máquinas remotas mediante conexiones seguras de red.
\bigskip

\textbf{SSL (Secure Sockets Layer)}: serie de protocolos criptográficos que proporcionan comunicaciones seguras por una red.
\bigskip

\textbf{R}: Entorno y lenguaje de programación con un enfoque al análisis estadístico. Se trata de uno de los lenguajes más utilizados en investigación por la comunidad estadística, siendo además muy popular en el campo de la minería de datos, la investigación biomédica, la bioinformática y las matemáticas financieras.
\bigskip

\textbf{Módulo}: fragmento de un programa desarrollado para realizar una tarea específica.
\bigskip

\textbf{Hash}: también llamadas funciones de resumen son algoritmos que consiguen crear a partir de una entrada (ya sea un texto, una contraseña o un archivo, por ejemplo) una salida alfanumérica de longitud normalmente fija que representa un resumen de toda la información que se le ha dado (es decir, a partir de los datos de la entrada crea una cadena que solo puede volverse a crear con esos mismos datos).
\bigskip

\textbf{Software libre}: software cuya licencia permite que este sea usado, copiado, modificado y distribuido libremente según el tipo de licencia que adopte.
\bigskip

% \endgroup


\newpage
\begin{thebibliography}{99}
	\addcontentsline{toc}{chapter}{Bibliografía}

% \subsubsection*{Libros consultados durante la realización del proyecto:}

% \printbibliography
\printbibliography[heading=bibempty]

\bigskip
\subsubsection*{Páginas de consulta sobre licencias, legislación y desarrollo y uso del software analizado}
\bibitem{CC} {\tt Creative Commons Share Alike 4.0}. \url{https://creativecommons.org/licenses/by-sa/4.0/}
\bibitem{rae} {\tt Diccionario RAE}. \url{http://dle.rae.es/}
\bibitem{boe} {\tt Boletín Oficial del Estado}. \url{https://www.boe.es/}
\bibitem{todofp} {\tt Todo FP}. \url{https://www.todofp.es/}
\bibitem{adide} {\tt ADIDE}. \url{https://www.adideandalucia.es/}
\bibitem{wikibooks} Wikibooks ({\tt LaTeX}). \url{https://en.wikibooks.org/wiki/LaTeX}
\bibitem{coderunner} {\tt Code Runner}. \url{https://github.com/trampgeek/moodle-qtype_coderunner/}
\bibitem{holamundo} {\tt Ejemplos del Hola Mundo}. \url{https://es.wikipedia.org/wiki/Anexo:Ejemplos_de_implementaci%C3%B3n_del_%C2%ABHola_mundo%C2%BB}
\bigskip
\subsubsection*{Otro material}
\begin{itemize}
	\item Diversas consultas puntuales al sitio {\tt Stack OverFlow}.
	\item Material docente de las asignaturas \textbf{Procesos y contextos educativos}, \textbf{Innovación docente e Investigación Educativa en Ciencia y Tecnología} y \textbf{Complementos de Formación} impartidas en el \master de la \textbf{Universidad de Granada}.
\end{itemize}
\end{thebibliography}





% \bibliographystyle{unsrt}
% \bibliography{project}

% \bibliographystyle{unsrt}
% \bibliography{bibliography/project}
% \addcontentsline{toc}{chapter}{Bibliografía}
% \bibliographystyle{plain}

\newpage \
\thispagestyle{empty}
\end{document}